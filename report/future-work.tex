\section{Future Work}
\label{sec:future-work}

There are several aspects of the application than can be improved. Which
can be categorized under one of the following: usability, and morphing.
The categories will be discusses individually as they are separated both
technically and conceptually. 

\subsection{Usability}

In terms of usability there has been no formal study on what aspects of
the application work work well. However informal testing has shown several
avenues of improvement. First is how to access the application. 

While the freely chosen app name and icon do suffice there are other ways 
it could be done. Namely the widgets available to android users. This could 
be a simple widget mimicking the standard available widgets, or even be 
invisible. So as to have no recognizable screen space. While it could still
react to certain use patterns. A first implementation of this functionality is
available but needs to be properly evaluated.

Once the application has been started there are still some ways the user
needs to be observant in how she uses the application. As it can result in
inadvertent data breaches. Mainly when the data is opened in other apps.
While several solutions have been discussed, like listening to home button
presses or having files only be unlocked for limited time, these still need
to be explored in depth.

\subsection{Morphing} 
\label{sec:limitations:morphing}

When it comes to morphing there are two major limitations; app renaming and software integrity validation. 
The first pertains to limitations in the renaming of the application when morphing. 
The second holds that there is no way to guarantee that the application hasn't been modified after morphing.

When it comes to naming there is a functional limitation in the length of the new name as of now.
It has to be the same length or shorter than the original name length, which can be padded with whitespace. 
But as of yet the way Android manifests are encoded are opaque enough that not all size indicators have been correctly identified.

The other component of naming is the `package name' staying consistent across morphs. 
The package name of an application is a naming convention which Android uses as an identifier for applications in its app store.
For casual searches, the package name is not relevant in most situations.
However there is one exception, namely if DroidStealth would be on an app store. 
As apps are identified there by package name this would make it easy to reveal its presence just by searching for it on the appstore.
Currently the only solution is to not be available through any app store.

The integrity validation is currently something that does not exist with morphing. 
The application needs to be signed with the same key before morphing as after morphing if users want to be able to reinstall it afterwards without first having to remove it. 
This makes it possible to distribute updates in a nomadic manner, but also means the signing key has to be packaged with DroidStealth for it to work. 
As such anyone can create a modified version of the application and spread that as an `update'.
Solving this problem is non-trivial and requires more research.
