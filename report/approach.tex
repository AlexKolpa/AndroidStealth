\section{Approach}
\label{sec:approach}
Here, we discuss the approach we have taken in creating DroidStealth. 
First, the concept of two-layered data obfuscation that we have applied to hide the data is explained.
After that, the applied techniques that allow nomadic distribution are discussed.

%% Here we say stuff about the libraries and intents because why reinvent the wheel. 
%% WHAT ELSE DO WE TALK ABOUT HERE EXACTLY BESIDES USER EXPERIENCE?
%% NOT SURE IF STRUCTURE HERE WORKS

\subsection{First Layer of Protection: Access Restriction}
The first layer of protection offered by DroidStealth is access restriction.
This restriction consists of two parts, one for the data itself and one for the application.

All data within DroidStealth is encrypted by default.
When the user adds a file to be managed by the application, it is automatically removed from its original location, encrypted, and added to a dedicated data folder.
This way, the data cannot be accessed by another application, and restricts users to opening these files through DroidStealth.
Even when the user wants to open an arbitrary file managed by DroidStealth, the conscious step of decrypting the file must be made first, raising awareness of the risk of exposure.

Every launch of the application requires the user to enter a pin code, which the user defines the first time DroidStealth is opened.
Only by entering this pin code upon launch can the user access the files managed by DroidStealth. Should it be forgotten, the data in the application will remain encrypted forever.

\subsection{Second Layer of Protection: Hiding DroidStealth}
The first layer of protection provides strong measures against casual inspectors trying to access the data or the application.
However, as explained in Section~\ref{sec:problem-description}, finding encrypted and inaccessible data or applications can raise suspicion towards the user.
Since simply encrypting the data is not enough, our approach provides an added step of obfuscation that increases security of the data: DroidStealth hides itself.
This combination provides the two-layered protection, which is the key solution implemented in DroidStealth.

With the purpose of hiding all incriminating aspects of the encrypted data, we apply two ways of hiding DroidStealth on the device: 
Providing alternative launching methods, and `hiding the application in plain sight' by changing DroidStealth's appearance. 
The first is employed if the user wants to hide DroidStealth from easy access on Android devices, while changing its appearance allows for easy access as well as protection from casual search.

\subsubsection{Alternative Launch Methods}
The default way of launching an application on an Android device is through the `app drawer'.
This overview of all installed applications would normally show the DroidStealth logo, and provide a way of opening it.
The application would then be visible to anyone browsing the app drawer, including a casual inspector.

It is possible to hide the application from this drawer, but the user still needs to be able to launch DroidStealth when it is hidden.
For this, we have implemented two alternative launching methods, which can be enabled by the user in the application.

\textbf{Dialer Launch}
DroidStealth provides a launch option that opens the application by dialing a user-defined numeric sequence. 
The user enters this numeric sequence in the any regular phone number dialing application, as it would with a phone number.
Instead of actually calling the number, the application launches, requesting the pin code. 
Furthermore, DroidStealth fully intercepts the call, making sure the number never gets added to the call log.

\textbf{Transparent Widget}
Another option is to launch the app by means of an invisible widget on the device's home screen. 
This widget takes the form of a transparent area, thus showing the background image, positioned on the home screen by the user.
When the widget area is tapped once or twice, it remains inactive and simply appears to be an empty area.
However, when the widget is tapped five times consecutively, it launches DroidStealth.

When first created, this widget is temporarily visible, to help the user in placing it and to remember the location of the widget.
After the first time the widget is tapped, it becomes transparent.

When adding a new access widget, all previously placed widgets become visible. 
This allows the user to retrieve forgotten widgets. 

\subsubsection{Morphing: Mimicking Other Applications}
DroidStealth is capable of transforming itself, changing its appearance as shown on the device.
The user can change the application's icon and name, and DroidStealth can then create an Android Application Package from which the application can be (re-)installed with the new appearance.
DroidStealth is also capable of listing all installed applications on the device, from which the user can easily pick the one it wants to mimic.

With the newly acquired appearance, DroidStealth does not need to be hidden from the app drawer and no secret launch methods are required.
This way, the intuitive `default' launching method is retained, while DroidStealth is still hidden from a casual inspector.
