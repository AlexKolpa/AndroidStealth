\section{Introduction}
\label{sec:introduction}
There are more than 1.5 billion smartphone users all over the world\cite{smartphoneUsage}, and over 80\% of the population in developed nations that have mobile broadband subscriptions~\cite{mobileBroadband}.
As such it is clear these mobile computing devices have become extremely popular in many countries.
Every day these devices are used to communicate, take pictures, game, and to facilitate many more aspects of daily life.
This results in smartphones taking over many functions that desktop and laptop computers have previously fullfilled.
It lets people carry with them everything they like, what they find important, pieces of their lives.
As a result there is a desire to control who has access to that data, and what that data is used for.

Humans are social beings and like to share, which smartphones build upon by making it easy to share the data electronically.
The choice to share in these cases is up to the user. 
This differs quite a bit from physically passing around a device.
Yet there are still a great many reasons to hand over a smartphone, such as letting someone take part in a phone call, letting them play a game, view vacation photographs.
After handing over the smartphone, the owner has very little control over what is done and how it gets used.
If that person does not want to rely on social norms there are very few options short of physically intervening in the handling of the device.
Such interventions are of limited use as the owner might not realize that their phone is being searched and an intervention may not be possible.
After all, the device's screen might not be in view of the owner at all times, or in some cases certain organisations may demand to search your phone.
For these kinds of situations it is desirebly to have the option of hiding some files like private photographs or sensitive documents.

One such situation arose with the arab spring where it became clear that such documents can have significant impact~\cite{arabSpring}.
Smartphone users provided on the ground and real time information of the protests and government reactions.
This allowed international media to use the material the people present at these events created to show close up views of these events in a manner that did not often happen before.
Yet these same regimes have not been standing still and ignoring these developments.
Countermeasures have been developed to stop the information from spreading like it has\cite{cyberResponseGovernment}.
This is before accounting for the effects of harsh punishments when someone is found having such data.
It is therefore a necessity to be able to hide data until it can be shared in a safe manner.

In this paper we will examine the current situation and existing approaches, and derive a new solution with an open source implementation for Android.
First we will define in detail the problem we provided the solution for.
Then we go into how we approached this problem and the ideas used to solve it.
This is followed by implementation details for the Android application.
Finally we describe potential improvements and further research.