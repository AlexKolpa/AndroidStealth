\section{Introduction}
\label{sec:introduction}
There are more than 1.5 billion smartphone users all over the world\cite{smartphoneUsage}, and over 80\% of the population in developed nations that have mobile broadband subscriptions~\cite{mobileBroadband}.
Every day these devices are used to communicate, take pictures, record movies, and capture daily life.
This results in smartphones taking over many functions that desktop and laptop computers have previously fullfilled.
It lets people carry with them everything they like, what they find important, pieces of their lives.
As a result there is a desire to restrict access to that data, and what that data is used for.

People love to share their daily lives with others, something which has been strongly facilitated by the use of smartphones.
The choice to share selected material in these cases is up to users themselves. 
This differs quite a bit from physically passing around a device where contents of a phone are fully exposed to an interested viewer.
Yet there are still a great many reasons to hand over a smartphone, such as letting someone take part in a phone call, letting them play a game, view vacation photographs.
After handing over the smartphone, the owner has very little control over what is done and how it gets used.
For these kinds of situations it is desirable to be able to hide some files like private photographs or sensitive documents.
Of course, much more serious situations are possible as well.

One such situation arose with the Arab Spring where it became clear that such documents can have a significant impact~\cite{arabSpring}.
Smartphone users provided on the ground real time information of the protests and government reactions through various social media platforms.
This allowed international media to show close up views of these events in a manner that did not often happen before.
However, most governments have not been ignoring these changes.
Nowadays, many governments actively monitor social media platforms or even block their usage~\cite{cyberResponseGovernment}.
Because of this, spreading content directly has become a popular approach in places where conventional platforms are either locked down or under close surveillance.
However, this means that people carry with them sensitive material, which if found by officials could lead to small to severe repercussions.
For many of these people, it is therefore a necessity to be able to hide data until it can be shared in a safe manner.

In this paper we will examine the current situation of data hiding on mobile devices, and derive a new solution with an open source implementation for Android.
For this, the existing problem of data hiding and sharing on mobile devices needs to be analyzed first.
Based on this analysis, a possible solution for the problem can then be derived.
Using this solution, we can describe the implementation of the solution for Android devices.
Finally, some potential improvements and further research will be discussed.
