Smartphones are increasingly common all over the world.
Every day they are used to communicate, take pictures, game, and to facilitate many more aspects of daily life.
Taking over many functions that desktop and laptop computers have previously fullfilled.
It lets people carry with them everything they like, what they find important, pieces of their lives.
As a result there is a desire to control who has access to that data, and what that data is used for.

Humans are social beings and like to share, which smartphones build upon by making it easy to share the data electronically.
The choice to share in these cases is up to the user. 
This differs quite a bit from physically passing around a device.
Yet there are still a great many reasons to hand over a smartphone, such as letting someone take part in a phone call, letting them play a game, view vacation photographs.
After handing over the smartphone, the owner has very little control over what is done and how it gets used.
If that person does not want to rely on social norms there are very few options short of physically intervening in the handling of the device.
Such interventions are of limited use as the owner might not realize that their phone is being searched and an intervention may not be possible.
After all, the device's screen might not be in view of the owner at all times, or in some cases certain organisations may demand to search your phone.
For these kinds of situations it is desirebly to have the option of hiding some files like private photographs or sensitive documents.

%TODO Determine if a paragraph with anecdotes for why hiding data is desireable is needed here.

In this paper we will describe one solution for accessible hiding of data, including an open source implementation of that solution for Androids
\textbf{Bla bla bla structure of paper and what we're going to discusswhere.}