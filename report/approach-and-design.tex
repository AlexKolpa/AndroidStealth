\section{Approach and Design}
\label{sec:approach-and-design}

Simply hiding a user's data won't be enough to pass the casual search that was mentioned earlier. 
Anyone inspecting a phone that is aware of our application and its functionality would become highly suspicious of any user with the application installed, resulting in more thorough inspection. 
This is something this application was not designed for, which could possibly result in the hidden data becoming compromised. 

From this it follows that not only the data needs to be hidden, but the application itself as well. 
Unfortunately, the Android framework does not provide many tools for this, mainly since the operating system, like most mobile operating systems, is focused on easy application access. 
This meant some unconvential approaches had to be assessed.

There are several obstacles in casual search for the application on Android. 
The most obvious obstacle is the so called 'app drawer'. 
This is a collection where the operating system shows all applications a user has installed.
While it is possible to change the way the app drawer behaves -- since it in itself is an application on Android as well -- changing the app drawer is a radical approach, which many users, especially those not technically literate, would not consider doing.
Since the product was meant to be as unobtrusive as possible to install and use, it meant that having an application that behaved as a custom app drawer was not an option.

The approach that was opted for was one of simplicity in concept, namely based on the idea of 'hiding a tree in the forest'. 
The application would have to be so unnoticable between the user's other applications, casual search would not expose it immediately. 
This proved another problem, since applications have a name and an icon that stays the same as long as they are not updated by the developer themselves. 
Updating continuously would not be an option, since the application ist meant for offline usage. 

Since the developers could not update the appearance of the application, this would be something the users would need to be able to do.
To achieve this, a novel approach was taken.
For this project a method was devised to allow users to alter the name and icon of the application without developer intervention, a method which was appropriately dubbed 'morphing' due to the now fluid behaviour application appearances can have.
Unfortunately, due to limitations in Android, users would need to re-install the application after having 'morphed' it, but this problem is addressed through the way users share the application.

As expressed in the section \ref{sec:problem-description}, the application should be able to be shared nomadically, meaning there is no central distribution point, but users share the application between each other. 
Combined with the morphing discussed in the previous paragraph, this provides for a good solution for  the necessity  of reinstalling the application for it to be morphed: 
Users can create a morphed version which they can share with other users, so that the application never looks the same, making it near impossible to detect during a casual search.

Sharing the application can be achieved through a multitude of ways on Android, but because of the ease of use and the increasing implementation, Near Field Communication (NFC) was chosen the main approach for data sharing. 
It is also possible to share through the default sharing system supported by Android -- which includes BlueTooth and uploading files to various sharing services -- for those devices that do not yet support NFC.