\subsection{Morphing}
\label{sec:implementation:morphing}

As described in section \ref{sec:approach-and-design}, to hide the application some way is needed to alter the appearance of the application so that it can be hidden from casual search.
This section will detail the implementation of technique, which has been dubbed `morphing'.
To achieve a complete change of appearance for an application in the app drawer, both its icon and its name need to be changed. 
However, to explain the full approach, some background on the inner workings of Android application packages might be required.

Android uses its own naming for its packages, so called Android Packages, or `.apk' files.
In reality, these files are very similar to Java archives -- so called `.jar' files -- namely that they are both archives containing executables and resources.
To construct an apk file, one first needs the executables and the resources (images, animations and layouts among other things), which can then be compressed into an archive.
Optionally, the archive can be zipaligned, which improves its read performance by reordering the contents of the package as to optimize it for Android devices.
Once the archive has been constructed, it only needs to be signed with an appropriate key for the Android system to accept it as a valid package.

For the morphing to be successful, this process needed to be reversed, and then repeated after altering the archive resources.
For this, the original package is required. 
Fortunately, Android allows access to the original package from within the application without root access (see section \ref{sec:problem-description} for an explanation about what root access means).
Reversing of the package construction process, extraction of the files, can be achieved through normal zip extraction, something which is included by default in the Java version used for Android.
Once the files have been extracted, the application resources can be altered.

As explained earlier, both the application icon and the application name need to be altered to have a successful morphed application.
To alter the icons, first the original icon name that is stored in the resources is extracted from the application info, something which Android provides through its API.
Then it is a matter of iterating over the extracted content to find the appropriate icon files -- Android allows for multiple resolutions of the same image to be stored -- and replace them with the user specified icon.
Once the icons have been replaced, the name of the application needs to be changed.
Unfortunately, this is where the first restriction imposed by Android is encountered.

Android uses a resource map where each resource item gets mapped to a unique id generated by the compiler, which allows for easy re-use of resources in the making of an application.
This does pose an obstacle, since these reference maps are compiled into a binary format which is difficult to alter.
There is a solution available for desktop environments\cite{website:apktool}, but it relies on Android's Package Tool, `AAPT'.
Porting AAPT to Android proved to be near impossible due to its system requirements, which meant that decompiling the Android resources would not be a feasible approach.
Fortunately, the name of the application is accessed through a file known as the Android Manifest, which contains general information about the application package.
This manifest proved to be slightly more processable from within the confinements of Android.
This meant that the new application name could be put directly into the manifest, without having to rely on the decompiling of Android resources.
After these two steps of altering the application appearance, the contents can be reconstructed again into a valid Android Package.

The first step is rebuilding the archive. 
Since it is structurally the same as a Java Archive, existing tools could be used. 
For this, the JarBuilder by Dominik Werthmueller\cite{website:jarbuilder} was used, since it posed the least amount of dependencies, which is favorable when working with Android, which can be rather picky about what parts of Java are actually supported in its system.
This resulted in a complete archive, which still needed to be zipaligned and signed with an appropriate key.
Because of similar restrictions posed by the decompiling of the Android resources, it was decided that including zip alignment was not achievable for now.

Finally, the package needs to be signed. 
Fortunately, an existing standalone solution is available outside the default Android signing methods, the `zip-signer' library\cite{website:zip-signer}.
However, a signing key still needs to be chosen. 
For the scope of this project, it was decided the test key would be sufficient, since actually signing it with appropriate keys which would needed to be tracked to prevent falsification of the application provide several challenges which will be discussed in section \ref{sec:limitations:morphing}.

Once the archive has been signed the morphing has been completed. 
The user can now be presented with an application package which contains the original application, albeit with a new appearance, according to the user's preferences.
This application can then be shared with other users.


