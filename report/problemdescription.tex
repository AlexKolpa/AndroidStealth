\section{Problem Description}
\label{sec:problem-description}
Using mobile phones as a primary platform for computing, recording data and storing data is becoming more and more ubiquitous.
While the users store sensitive data on these devices the security measures to protect this data often prove to be insufficient.
This can be because of the implementation but more often these measures are hard to apply properly.
Even the more popular security measures available, such as PIN codes, `unlock patterns' and passwords have proven not to be effective at all times.
There are even techniques to retrieve these codes and patterns, for example using smudge attacks\cite{aviv2010smudge}.

This section describes the problem we attempt to solve.
It does so by explaining the sensitive data that should be kept safe, the situation (environment) in which it should provide protection, and limitations that arise. 
First it will go into what kind of data it should involve and how to handle it. 
Then an explanation of what this data should be protected against. 
Followed by a section on 'root access' on a phone, what that means, and why that can not be required.
Finally we discuss the need for nomadic distribution.

\subsection{Hiding and Securing Data}
Sensitive data, e.g. documents, photos, and other kinds of files that the user wants to keep private, can take many forms.
The data will likely consist primarily of photos and videos with the occasional document but should not exclude other types of file.
As such the application should support the file types that the device could normally handle.

An obvious solution to keep data safe from undesired inspection is encryption made it a must.
Encryption, properly implemented, would provide a way of securing the data by making it impossible to understand without the key. 
This has the added benefit of hiding the encrypted data by making it unrecognizable as the file it once was.
Preventing many applications on a smart phone from showing them to a user, though not all.

Only hiding the data is not enough however.
When the application is spotted and its purpose is know this would cause suspicion and is likely to result in follow up measures.
The application was thus made to not be easy to find or recognizing, further increasing the chance of passing a casual search.

\subsection{Casual Search}
As discussed in the introduction, we attempt to address the problem of `casual search'. 
Before we continue however the term will be defined formally below, as the introduction only provides a first idea of the concept.
In the scope of this article, we define `casual search' as following: A quick attempt to find information on a mobile device without applying advanced technical knowledge of the device, nor specific knowledge of the protecting methods proposed and implemented as described in this article.

We aim to protect against this investigation method, as it is likely to be the first investigation encountered; pass it and the person in question will likely not searched more thoroughly by a specialist or someone with more knowledge.
Furthermore, casual search is probably the most common as more intensive searches usually require very limited resources.
After all a `quick glance' through a phone, be it at a checkpoint, a protest, or because your significant other has gotten hold of the device is more likely to occur than a thorough investigation of the device by someone who is experienced and trained where to look.

\subsection{No root}
The solution that is described in this article had a technical limitation that put a certain set of tools out of reach.
`Rooting' an Android device unlocks administrative privileges, allowing the user or applications to access restricted files, and communicate with device features more directly.
By doing this, it undermines Android's security model\cite{vidas2011all}.

Often named `root access', it can be compared with root access as in Unix systems, and Administrator privileges in Windows systems.
Because `root access' allows manipulating system files, it provides more powerful tools that could benefit the solution to the problem addressed.

However, most users do not have a rooted phone, nor do do they have the technical knowhow to root their phone themselves. 
Rooting a phone has significant disadvantages if done incorrectly; device's software could get corrupted to the point where it needs to be completely reset or becomes unusable.
Users will also be hesitant to root their phone because it voids the warranty.
To keep the solution available for all Android phones, the use of root access is not an option, and can therefore not be applied.

\subsection{Nomadic distribution}
In situations where a central distributor of applications is monitored or even censored, distribution of the application can prove difficult.
In these cases, the application we propose in our solution is likely to be blocked.
Even if it is not, knowledge of the application will quickly reach those who desire access to the data the users want to keep private.

A way of circumventing this could be a `nomadic' distribution of the application.
Instead of providing the application through a central point, that can be monitored, the users could share it directly with those interested.
More importantly, they can share it only with those they trust.
This provides extra means of keeping the application in the right hands, and thus preventing the knowledge of the application entirely.  