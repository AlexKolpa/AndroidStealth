\section{Problem Description}
\label{sec:problem-description}
Using mobile phones as a primary platform for computing, recording data and storing data is becoming more and more ubiquitous.
While the users store sensitive data on these devices, the security measures to protect this data, if even applied, often prove to be insufficient.
Even the more popular security measures available, such as PIN codes, `unlock patterns' and passwords have proven not to be effective at all times.
There are known techniques to retrieve these codes and patterns; for example, using smudge attacks\cite{aviv2010smudge}.

This section describes the issues we address, and hope to solve with DroidStealth.
It does so by explaining the sensitive data that should be kept safe, the situation (environment) in which it should provide protection, and limitations that arise.

\subsection{Audience}
The audience of the application is broadly defined.
DroidStealth's inspiration originates from the `citizen journalism'\cite{duffy2011} that played a significant role in the Arab Spring. 
As such a worst case of people dealing with an oppressive society is taken into account.
The worst case does not define the entire audience though.
Anyone interested in hiding and securing data on their device is part of the target audience for this application.

\subsection{Casual Search}
This application addresses the problem of `casual search'.
In the scope of this article, we define `casual search' as following: 
"A quick attempt to find information on a mobile device without applying advanced technical knowledge of the device, nor specific knowledge of the protecting methods proposed and implemented as described in this article".
Someone applying a casual search is named a casual inspector, or casual searcher.
We aim to protect against this investigation method, as it is likely to be the first investigation encountered; pass this and you will likely not be subjected to more intensive searches at all.
This reasoning also assumes that casual search is the most used search method:
A `quick glance' through a mobile device is more likely to occur than a thorough investigation of the device by someone who is experienced and trained where to look.
Examples of casual search have occurred during the protests in the Arab Spring, when related governments have inspected mobile devices of protesters leaving a demonstration.
\todo{I don't have a source for the last sentence.}

\subsection{Beyond Simple Encryption}
An obvious solution to keep data safe from inspection by the wrong audience is encrypting the data.
Using a secure encryption has two useful properties:
First, the data would obviously become inaccessible to any inspector.
Second, the encryption would render the files unreadable and unrecognizable to a human inspector, thus making them less incriminating.

However, using encryption makes you suspicious.
The inaccessible data can be a reason to extend a casual search with more intensive search methods, in which case the data might be compromised after all.
The same applies to the application that holds the data; any restriction to opening the application can be seen as suspicions.
To address both issues, the application should not only hide the data, but itself as well.

In this article, we will use the term `locked data' to refer to encrypted, inaccessible data.
With `locked application' we mean the application with its access restriction, as we will explain in Section~\ref{sec:approach}.

\subsection{Root Permissions}
The solution that is described in this article had a technical limitation that put a certain set of tools out of reach.
`Rooting' an Android device unlocks administrative privileges, allowing the user or applications to access restricted files, and communicate with device features more directly.
By doing this, it undermines Android's security model\cite{vidas2011all}.

Often named `root access', it can be compared with root access as in Unix systems, and Administrator privileges in Windows systems.
Because `root access' allows manipulating system files, it provides more powerful tools that could benefit the solution to the problem addressed.

However, most users do not have a rooted phone, and are not capable of rooting it either. 
Rooting a phone has significant disadvantages; one could damage the device's software to the point where it needs to be completely reset.
Users will also be hesitant to root their phone because it voids the warranty.

To keep the solution available for all Android phones, the use of root access is not an option, and can therefore not be applied.

\subsection{Nomadic distribution}
In situations where a central distributor of applications is monitored by those from whom the data has to be protected, distribution of the application can prove difficult.
In these cases, the application we propose in our solution is likely to be blocked.
Even if it is not, knowledge of the application will quickly reach those who want to access the data the users want to keep to themselves.

A way of circumventing this could be a `nomadic' distribution of the application.
Instead of providing the application through a central point, that can be monitored, the users could share it directly with those interested.
This provides extra means of keeping the application in the right hands, and thus preventing the knowledge of the application entirely.  

\subsection{Data Type Restrictions}
The sensitive data that the user wants to keep safe can take many forms.
The data managed in DroidStealth will likely consist mostly of photos and videos, with the occasional document. 
However, we cannot predict exactly what data the user would want to hide in the application.
Therefore, the application should support the file types that the device could normally handle, since we do not want to restrict the user to certain file types.
