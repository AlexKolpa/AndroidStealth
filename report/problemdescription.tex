A way to achieve that for casual search is by hiding the data on the device.
Preferably as it is being made. If it is not in places where it can be seen
by looking over the device it preserves data integrity, and by making it
only accessible to those who know the secret. Either of the data location
and of encryption key.

Of course this should minimally impede the ability to access and share the
data. Assuming that Internet access is limited or monitored. Direct transfer
methods like bluetooth, wi-fi direct, and of course sharing to other apps,
can be utilized to achieve this.

\section{Problem Description}
\label{sec:problem-description}
Using mobile phones as a primary platform for computing, recording data and storing data is becoming more and more ubiquitous.
While the users store sensitive data on these devices, the security measures to protect this data, if even applied, often prove to be insufficient.
Even the more popular security measures available, such as PIN codes, `unlock patterns' and passwords have proven not to be effective at all times.
There are even techniques to retrieve these codes and patterns, for example using smudge attacks\cite{aviv2010smudge}.

This section describes the problem we attempt to solve.
It does so by explaining the sensitive data that should be kept safe, the situation (environment) in which it should provide protection, and limitations that arise.
In Section \ref{sec:approach-and-design}, the approach to the solution is described. 

\subsection{Hide and encrypt}
Sensitive data, which data the user wants to keep safe in general, can take many forms.
The data will likely consist mostly of photos and videos, with the occasional document; regardless, the application should support the file types that the device could normally handle.

The data should be made inaccessible by hiding and encrypting it.

WIP
\subsection{Casual Search}
As discussed in the introduction, we attempt to address the problem of `casual search'
\footnote{In the scope of this article, we define `casual search' as following: A quick attempt to find information on a mobile device without applying advanced technical knowledge of the device, nor specific knowledge of the protecting methods proposed and implemented as described in this article.}.
We aim to protect against this investigation method, as it is likely to be the first investigation encountered; pass this and you will likely not be subjected to more intensive searches at all.
Furthermore, casual search is probably the most used search method. 
A `quick glance' through a phone, be it at a checkpoint, a protest, or because your significant other has gotten hold of the device is more likely to occur than a thorough investigation of the device by someone who is experienced and trained where to look.

\subsection{WiFi Hotspot}
WIP
\subsection{No root}
The solution that is described in this article had a technical limitation that put a certain set of tools out of reach.
`Rooting' an Android device (somewhat equivalent to `Jailbreaking' an Apple device) unlocks administrative privileges, also called `root access'.
This `root access' allows manipulating system files, which in turn provides more powerful tools that could benefit the solution to the problem addressed.

However, most users do not have a rooted phone, and are not capable of rooting it either. 
Rooting a phone has significant disadvantages; one could damage the device's software to the point where it needs to be completely reset.
Users will also be hesitant to root their phone because it voids the warranty.

To keep the solution available for all Android phones, the use of root access is not an option, and can therefore not be applied.
\subsection{No unmonitored app distributor}
Google Play is not available in for example Iran.

WIP
