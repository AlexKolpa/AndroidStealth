\documentclass[twocolumn,english,compsoc,journal]{IEEEtran}
\usepackage[T1]{fontenc}
\usepackage{babel}
\usepackage[unicode=true,
 bookmarks=true,bookmarksnumbered=true,bookmarksopen=true,bookmarksopenlevel=1,
 breaklinks=false,pdfborder={0 0 0},backref=false,colorlinks=false]
 {hyperref}
\hypersetup{pdftitle={Your Title},
 pdfauthor={Your Name},
 pdfpagelayout=OneColumn, pdfnewwindow=true, pdfstartview=XYZ, plainpages=false}
\usepackage{breakurl}

\makeatletter

%% Because html converters don't know tabularnewline
\providecommand{\tabularnewline}{\\}

 % protect \markboth against an old bug reintroduced in babel >= 3.8g
 \let\oldforeign@language\foreign@language
 \DeclareRobustCommand{\foreign@language}[1]{%
   \lowercase{\oldforeign@language{#1}}}

% for subfigures/subtables
\usepackage[caption=false,font=normalsize,labelfont=sf,textfont=sf]{subfig}
%\usepackage[nocompress]{cite} %optional

\makeatother

\begin{document}



\title{DroidStealth: A nomadic data obfuscation tool that facilitates sharing}


\author{Olivier Hokke, Alex Kolpa, Joris van den Oever and~Alex Walterbos}


\markboth{Delft University of Technology Student Project}{Your Name \MakeLowercase{\textit{et al.}}: Your Title}

\IEEEtitleabstractindextext{
\begin{abstract}
TODO ABSTRACT TEXT\end{abstract}

\begin{IEEEkeywords}
casual search, privacy, nomadic software, obfuscation
\end{IEEEkeywords}

}

\maketitle

\IEEEdisplaynontitleabstractindextext{}


\IEEEpeerreviewmaketitle{}


\section{Introduction}

\IEEEPARstart{W}{ith} rising use of smart phones in daily life and
exceptional events means there is a lot more data available on phones.
These pictures, videos, and in some cases other files are very interesting
to a wide variety of groups. This might be because oppresive governments
want to suppress this data at checkpoints around protests or random
searches on the street, or even individuals with authority pressuring
to hand over the phone to grab/remove data themselves. There are many
instances of law enforcement officers oversteppin their legal bounds
to remove images of actions they don't want recorded or because of
mistaken ideas about the law. 

We call such non-technical searches of devices 'casual search.' The
person performing it may have varying degrees of expertise regarding
the workings of mobile computing devices but is limited to the tools
already available on the phone to check out data. Instead of performing
advanced attacks using specialised tools designed to get all the data
out of a phone. 

This project aims to address only the casual searches with an Android
app rather than the latter. Which is nearly impossible to defend against
if enough time and effort is expended. This is done by providing more
control over private data in these situations. It allows users to
lock and hide their content, without inhibiting the sharing whenever
they have the desire to do so. Yet remain difficult to detect even
to those who know of its existence.

\textbf{Bla bla bla structure of paper and what we're going to discuss
where.}


\section{Problem Description}

Describe exactly what we are trying to solve in detail.


\subsection{subsection}


\subsection{another subsection}


\section{Design and software architecture}

Here we say stuff about the libraries and intents because why reinvent
the wheel.

Data sharing and application sharing.


\section{Implementation}

How did we create the data vault. Encryption, thumbnails, notifications
for unlocking.

What the hell do we do to make morphing possible.


\section{Future Work}

bla bla, unlimited naming for morphed apps. (UTF-8, length limits?
Byte length indicator start of manifest and length of name field.)

Sharing, invisible widget

Locking files on timer or detecting if app that was opened on intent
is closed/home buttoned. Since that leaves stuff open.


\appendices{}


\section{Appendix citing}

Citation: \cite{example:beebe_archive}


\section{Appendix example for tables and images.}

\begin{figure}[htbp]
\begin{centering}
\textsf{A single column figure goes here}
\par\end{centering}

\protect\caption{Captions go \emph{under} the figure}
\end{figure}
\begin{table}[htbp]
\protect\caption{Table captions go \emph{above} the table}


\centering{}%
\begin{tabular}{|c|c|}
\hline 
delete & this\tabularnewline
\hline 
\hline 
example & table\tabularnewline
\hline 
\end{tabular}
\end{table}



\section*{Acknowlegment}

bla bla bla Pauwelse?

\bibliographystyle{IEEEtran}
\bibliography{references}

\end{document}
