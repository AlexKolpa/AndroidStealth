\section{Approach and Design}
\label{sec:approach-and-design}
\todo{Add Two-Layer stuff}
This section walks though all design choices made with DroidStealth,
and how we approached the problem in order to tackle it. Most 
solutions have drawbacks, which will be discussed further in the
section about future work and limitations. %% ADD REFERENCE

%% Here we say stuff about the libraries and intents because why reinvent the wheel. 
%% WHAT ELSE DO WE TALK ABOUT HERE EXACTLY BESIDES USER EXPERIENCE?
%% NOT SURE IF STRUCTURE HERE WORKS

\subsection{Morphing}
\label{sec:approach-and-design:morphing}

\todo{Move elsewhere.}
Simply hiding a user's data will not be enough to pass a casual search. 
Anyone inspecting a phone that is aware of our application and its functionality would become highly suspicious of any user with the application installed, resulting in more thorough inspection. 
This is something this application was not designed for, which could possibly result in the hidden data becoming compromised. 

From this it follows that not only the data needs to be hidden, but the application itself as well. 
Unfortunately, the Android framework does not provide many tools for this. 
This is mainly because the operating system, like most mobile operating systems, is focused on easy application access instead of easy file access. 
This meant some unconvential approaches had to be assessed.

There are several obstacles in casual search for the application on Android. 
The most obvious obstacle is the so called `app drawer', which is an overview of all the applications installed on the Android device.
In Android is possible to change the way the app drawer behaves, since it in itself is an application as well.
However, changing the app drawer is a radical approach, which many users, especially those not technically literate, would not consider doing.
Since the product was meant to be as unobtrusive as possible to install and use, it meant that having an application that behaved as a custom app drawer was not an option.

This meant that the application would have to be able to hide itself.
It would have to be so unnoticable between the user's other applications, casual search would not expose it immediately. 
This proved another problem, since applications have a name and an icon that stays the same as long as they are not updated by the developer themselves. 
Updating continuously would not be an option, since the application is meant for offline usage, as explained in section \ref{sec:problem-description}. 

Since the developers can not update the appearance of the application, this means it is something the users would need to be able to do themselves.
To achieve this, a novel approach was taken.
For this project a method was devised to allow users to alter the name and icon of the application without developer intervention, a method which was appropriately dubbed `morphing' due to the now fluid behaviour application appearances can have.
Unfortunately, due to limitations in Android, users would need to re-install the application after having `morphed' it, but this problem is addressed through the way users share the application.

As expressed in the section \ref{sec:problem-description}, the application should be able to be shared nomadically, meaning there is no central distribution point, but users share the application between each other. 
Combined with the morphing discussed in the previous paragraph, this provides for a good solution for the necessity of reinstalling the application for it to be morphed: 
Users can create a morphed version which they can share with other users, so that the application never looks the same, making it near impossible to detect during a casual search.

Sharing the application can be achieved through a multitude of ways on Android, but because of the ease of use and the increasing implementation, Near Field Communication (NFC)\cite{website:nfc-spec} was chosen the main approach for data sharing. 
It is also possible to share through the default sharing system supported by Android -- which includes BlueTooth and uploading files to various sharing services -- for those devices that do not yet support NFC.

\subsection{Styling}

DroidStealth is themed with a dark color in order to give users the feeling that they are indeed
working in secret, and that their data will be safely hidden. However, the app should not
have an amateuristic feeling about it. This would indicate that the app is potentially unstable, 
since it would give the user the sense that the developers had no sense of perfection. 
Therefore, we used a rounded, but solid straight font for our titles, and a formal font for 
all other texts. Combining this with the use of bright, but slightly softened primary and 
secondary colors, the user is given the feeling that the app handles its tasks well, but in 
the mean time, the user is soothed with a comfortable look. %% TODO: ADD SMALL IMAGE

The use of green and red is mirrored from the conventional meaning of those colors. Green 
will always be used to indicate that something is good or safe, red will do the opposite. 
In addition we have the color orange, which is used to indicate progress. Whenever a file is 
locking, or unlocking, the status bar of the file will be orange, and a animated twirl is shown 
to indicate that the file is being processed. %% TODO: ADD SMALL IMAGE

Finally, the gallery uses no padding or margins between the thumbnails of files, as we want 
to optimally make use of all the screen space. Thumbnails are thus as big as they can be, 
and thus will be most efficient in showing the user its contents. 
