\subsection{The Encryption Service}
\label{sec:implementation:encryption}
DroidStealth uses a `Service'\footnote{An Android Service is a part of an application that runs in the background, often used for more intensive or lengthy executions.} for the encryption of files.
This service runs in the background, and listens for \texttt{Intents} started by the application.
A queue is used to order the requests, and the service running in the backgrounds works through the queue continuously.

The encryption that DroidStealth provides is implemented using Facebook's Conceal API\cite{facebookConceal}.
Conceal provides a set of APIs for data encryption and authentication; we only use the first.
The Conceal library does not implement cryptography algorithms, but instead uses algorithms used in OpenSSL\cite{openssl}.

Files are encrypted individually.
When encryption would be applied at folder level, the application would have to decrypt all data upon being opened.
Not only does this expose all data during the user's interaction with (most likely) only one of the files;
but if the user would then forget to lock the data, all data managed by DroidStealth would remain exposed.
By applying per-file encryption, the risk of exposure is kept to a minimum.
Only loading files when needed also means an increase in startup time, as it is not necessary to decrypt the entire folder.

The process of encrypting an unencrypted file is quite simple because of the use of the Conceal API:
The \texttt{Crypto} class, provided by the Conceal API, handles all encryption logic in the process.
A `plain' Java input stream is created from the unencrypted file, and the \texttt{Crypto} class provides an output stream that encrypts data as it is being passed through.
When both streams have been created, a dedicated algorithm copies the data, buffered in chunks of 4096 bytes, from the unencrypted file's input stream to the encrypting output stream.
The final result is, as expected, the encrypted version of the previously unencrypted file.
The decryption process uses the exact opposite method:
An input stream provided by the \texttt{Crypto} class provides the decryption logic, by which the encrypted file is read.
The output of that stream is passed to a plain Java output stream, which writes the data to a new, unencrypted file.

The files are then stored in a folder outside the application folder.
To allow the updating of the application without data loss, this separation is required; overwriting the application folder may be required, especially when installing a morphed version of DroidStealth.
This means that the folder storing the files is public; other applications, as well as the user, can find it on their device via a computer or with a file explorer app.
The risk posed by this is migitated by the encryption of the files, as well as the requirement to know of DroidStealth, transcending the scope of `casual search'.
