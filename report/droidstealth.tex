\documentclass[twocolumn,english,compsoc,journal]{IEEEtran}
\usepackage[T1]{fontenc}
\usepackage{babel}
\usepackage[unicode=true,
 bookmarks=true,bookmarksnumbered=true,bookmarksopen=true,bookmarksopenlevel=1,
 breaklinks=false,pdfborder={0 0 0},backref=false,colorlinks=false]
 {hyperref}
\hypersetup{pdftitle={Your Title},
 pdfauthor={Your Name},
 pdfpagelayout=OneColumn, pdfnewwindow=true, pdfstartview=XYZ, plainpages=false}
\usepackage{breakurl}
\usepackage{color}

\makeatletter

%% Because html converters don't know tabularnewline
\providecommand{\tabularnewline}{\\}

 % protect \markboth against an old bug reintroduced in babel >= 3.8g
 \let\oldforeign@language\foreign@language
 \DeclareRobustCommand{\foreign@language}[1]{%
   \lowercase{\oldforeign@language{#1}}}

% for subfigures/subtables
\usepackage[caption=false,font=normalsize,labelfont=sf,textfont=sf]{subfig}

%\usepackage[nocompress]{cite} %optional

\newcommand{\todo}[1]{\colorbox{yellow}{#1}}

\makeatother

\begin{document}



\title{A Nomadic Self-Compiling Android Data Obfuscation Tool}


\author{Olivier Hokke, Alex Kolpa, Joris van den Oever, Alex Walterbos, and Johan Pouwelse (Course supervisor)}


\markboth{Delft University of Technology Student Project}{Your Name \MakeLowercase{\textit{et al.}}: Your Title}

\IEEEtitleabstractindextext{
\begin{abstract}
	Smartphones are becoming more significant for storing and transferring data, yet techniques making sure this data is not compromised in a search of the device are not readily available. 
	DroidStealth, an open source Android application, provides a smartphone user not only with means to encrypt the data on the device, but also hides itself through different obfuscation techniques. 
	This includes hiding the application's default launch methods and providing methods such as dial-to-launch and transparent launch buttons. 
	A novel technique provided by DroidStealth is the ability to morph the application to hide it in plain sight, using self-compilation, and without requiring root permissions.
	This Two-Layer protection aims to protect the user and its data from casual search in various situations.
\end{abstract}

\begin{IEEEkeywords}
casual search, 
privacy, 
nomadic software, 
data obfuscation,
data encryption
\end{IEEEkeywords}

}

\maketitle

\IEEEdisplaynontitleabstractindextext{}


\IEEEpeerreviewmaketitle{}


\section{Introduction}

\IEEEPARstart{W}{ith} the rising use of smart phones in daily life and
exceptional events means sensitive data is commonly available on phones.
These pictures, videos, and in some cases other files are very interesting
to a wide variety of groups. And sometimes the standard protections on
the phone are not enough to keep those groups out. Just slide the screen
and full access to everything. Even if there things like a password or
unlock patterns these can be forced out of a person. Followed by a simple
search through the phones files by hand to find what they desire.

We call such non-technical searches of devices 'casual search.' The
person performing it will have varying degrees of expertise regarding
the workings of smartphones but is limited to the tools already available
on the phone to check out data. Instead of performing advanced attacks
using specialized tools designed to get all the data out of a phone.

This project aims to address only the issues with casual search through
the use of an Android app. The tool-aided expert attacks are not within the
scope of this project.

This paper starts with explaining the problem description in section \ref{sec:problem-description}, wherein the problem that led to the setup of this project is expanded upon.
Next, the approach taken to this problem is discusssed in section \ref{sec:approach}.
Once the approach has been determined, a description of the actual implementation can be provided, which is done is section \ref{sec:implementation}.
Finally, the limitations and future work will be discussed in section \ref{sec:future-work}.

\section{Problem Description}
\label{sec:problem-description}
Using mobile phones as a primary platform for computing, recording data and storing data is becoming more and more ubiquitous.
While the users store sensitive data on these devices the security measures to protect this data often prove to be insufficient.
This can be because of implementation failures but often these measures are hard to apply properly by the average user\cite{usablesecurity,cannotsecurity}.
Even the more popular security measures available, such as PIN codes, `unlock patterns' and passwords have proven not to be effective at all times.
There are even techniques to retrieve these codes and patterns, for example using smudge attacks\cite{aviv2010smudge}.

This section describes the problem we attempt to solve.
It does so by explaining the sensitive data that should be kept safe, the situation (environment) in which it should provide protection, and limitations that arise. 
First it will go into what kind of data it should involve and how to handle it. 
Then an explanation of what this data should be protected against. 
Followed by a section on 'root access' on a phone, what that means, and why that can not be required.
Finally we discuss the need for nomadic distribution.

\subsection{Hiding and Securing Data}
Sensitive data, files that the user wants to keep private, can take many forms.
The data will likely consist primarily of photos and videos with the occasional document but should not exclude other types of file.
As such the application should support the file types that the device could normally handle.

An obvious solution to keep data safe from undesired inspection is encrypting it.
Encryption, properly implemented, would provide a way of securing the data by making it impossible to access without the encryption key. 
This has the added benefit of hiding the encrypted data by making it unrecognizable as the file it once was.
Preventing many applications on a smart phone from showing them to a user, though not all.

Yet hiding the data is not enough.
The presence of the app can raise suspicion, especially when its purpose is known. 
This could result in follow-up measures.
Thus the app should be hard to find or recognize, further increasing the chance of passing a casual search.

\subsection{Casual Search}
As discussed in the introduction, we attempt to address the problem of `casual search'.
In the scope of this article, we define `casual search' as following: A quick attempt to find information on a mobile device without applying advanced technical knowledge of the device, nor specific knowledge of the protecting methods proposed and implemented as described in this article.

We aim to protect against this investigation method, as it is likely to be the first investigation encountered; pass it and a more thorough search will be unlikely.
Furthermore, casual search is probably the most common as more intensive searches usually require very limited resources.
After all a `quick glance' through a phone, be it at a checkpoint, a protest, or because your significant other has gotten hold of the device is more likely to occur than a thorough investigation of the device by someone who is experienced and trained where to look.

\subsection{No root}
The solution that is described in this article had a technical limitation that put a certain set of tools out of reach.
`Rooting' an Android device unlocks administrative privileges, allowing the user or applications to access restricted files, and communicate with device features more directly.
By doing this, it undermines Android's security model\cite{vidas2011all}.

Often named `root access', it can be compared with root access as in Unix systems, and Administrator privileges in Windows systems.
Because `root access' allows manipulating system files, it provides more powerful tools that could benefit the solution to the problem addressed.

However, most users do not have a rooted phone, nor do do they have the technical knowledge to root their phone themselves. 
Rooting a phone has significant disadvantages if done incorrectly; the device's software could get corrupted to the point where it needs to be completely reset or becomes unusable.
Users will also be hesitant to root their phone because it voids the warranty.
To keep the solution available for all Android phones, the use of root access is not an option, and can therefore not be applied.

\subsection{Nomadic distribution}
In situations where a central distributor of applications is monitored or censored, distribution of the application can prove difficult.
In these cases, the application we propose in our solution is likely to be blocked.
Even if it is not, knowledge of the application will quickly reach those who desire access to the data the users want to keep private.

A way of circumventing this could be a `nomadic' distribution of the application.
Instead of providing the application through a central point, that can be monitored, the users could share it directly with those interested.
More importantly, they can share it only with those they trust.
This provides extra means of keeping the application in the right hands, and thus preventing the knowledge of the application entirely.  

\section{Approach}
\label{sec:approach}
Here, we discuss the approach we have taken in creating DroidStealth. 
First, the concept of two-layered data obfuscation that we have applied to hide the data is explained.
After that, the applied techniques that allow nomadic distribution are discussed.

%% Here we say stuff about the libraries and intents because why reinvent the wheel. 
%% WHAT ELSE DO WE TALK ABOUT HERE EXACTLY BESIDES USER EXPERIENCE?
%% NOT SURE IF STRUCTURE HERE WORKS

\subsection{First Layer of Protection: Access Restriction}
The first layer of protection offered by DroidStealth is access restriction.
This restriction consists of two parts, one for the data itself and one for the application.

All data within DroidStealth is encrypted by default.
When the user adds a file to be managed by the application, it is automatically removed from its original location, encrypted, and added to a dedicated data folder.
This way, the data cannot be accessed by another application, and restricts users to opening these files through DroidStealth.
Even when the user wants to open an arbitrary file managed by DroidStealth, the conscious step of decrypting the file must be made first, raising awareness of the risk of exposure.

Every launch of the application requires the user to enter a pin code, which the user defines the first time DroidStealth is opened.
Only by entering this pin code upon launch can the user access the files managed by DroidStealth. Should it be forgotten, the data in the application will remain encrypted forever.

\subsection{Second Layer of Protection: Hiding DroidStealth}
The first layer of protection provides strong measures against casual inspectors trying to access the data or the application.
However, as explained in Section~\ref{sec:problem-description}, finding encrypted and inaccessible data or applications can raise suspicion towards the user.
Since simply encrypting the data is not enough, our approach provides an added step of obfuscation that increases security of the data: DroidStealth hides itself.
This combination provides the two-layered protection, which is the key solution implemented in DroidStealth.

With the purpose of hiding all incriminating aspects of the encrypted data, we apply two ways of hiding DroidStealth on the device: 
Providing alternative launching methods, and `hiding the application in plain sight' by changing DroidStealth's appearance. 
The first is employed if the user wants to hide DroidStealth from easy access on Android devices, while changing its appearance allows for easy access as well as protection from casual search.

\subsubsection{Alternative Launch Methods}
The default way of launching an application on an Android device is through the `app drawer'.
This overview of all installed applications would normally show the DroidStealth logo, and provide a way of opening it.
The application would then be visible to anyone browsing the app drawer, including a casual inspector.

It is possible to hide the application from this drawer, but the user still needs to be able to launch DroidStealth when it is hidden.
For this, we have implemented two alternative launching methods, which can be enabled by the user in the application.

\textbf{Dialer Launch}
DroidStealth provides a launch option that opens the application by dialing a user-defined numeric sequence. 
The user enters this numeric sequence in the any regular phone number dialing application, as it would with a phone number.
Instead of actually calling the number, the application launches, requesting the pin code. 
Furthermore, DroidStealth fully intercepts the call, making sure the number never gets added to the call log.

\textbf{Transparent Widget}
Another option is to launch the app by means of an invisible widget on the device's home screen. 
This widget takes the form of a transparent area, thus showing the background image, positioned on the home screen by the user.
When the widget area is tapped once or twice, it remains inactive and simply appears to be an empty area.
However, when the widget is tapped five times consecutively, it launches DroidStealth.

When first created, this widget is temporarily visible, to help the user in placing it and to remember the location of the widget.
After the first time the widget is tapped, it becomes transparent.

When adding a new access widget, all previously placed widgets become visible. 
This allows the user to retrieve forgotten widgets. 

\subsubsection{Morphing: Mimicking Other Applications}
DroidStealth is capable of transforming itself, changing its appearance as shown on the device.
The user can change the application's icon and name, and DroidStealth can then create an Android Application Package from which the application can be (re-)installed with the new appearance.
DroidStealth is also capable of listing all installed applications on the device, from which the user can easily pick the one it wants to mimic.

With the newly acquired appearance, DroidStealth does not need to be hidden from the app drawer and no secret launch methods are required.
This way, the intuitive `default' launching method is retained, while DroidStealth is still hidden from a casual inspector.


\section{Implementation}
\label{sec:implementation}
\todo{Open source.}

This section will discuss the implementation of the application in detail. 
We first discuss the implementation of the `morphing' capabilities, after which the encryption service is explained.
\todo{Reread paragraph.}

\subsection{Morphing}
\label{sec:implementation:morphing}

As described in section \ref{sec:approach-and-design}, to hide the application some way is needed to alter the appearance of the application so that it can be hidden from casual search.
This section will detail the implementation of technique, which has been dubbed `morphing'.
To achieve a complete change of appearance for an application in the app drawer, both its icon and its name need to be changed. 
However, to explain the full approach, some background on the inner workings of Android application packages might be required.

Android uses its own naming for its packages, so called Android Packages, or `.apk' files.
In reality, these files are very similar to Java archives -- so called `.jar' files -- namely that they are both archives containing executables and resources.
To construct an apk file, one first needs the executables and the resources (images, animations and layouts among other things), which can then be compressed into an archive.
Optionally, the archive can be zipaligned, which improves its read performance by reordering the contents of the package as to optimize it for Android devices.
Once the archive has been constructed, it only needs to be signed with an appropriate key for the Android system to accept it as a valid package.

For the morphing to be successful, this process needed to be reversed, and then repeated after altering the archive resources.
For this, the original package is required. 
Fortunately, Android allows access to the original package from within the application without root access (see section \ref{sec:problem-description} for an explanation about what root access means).
Reversing of the package construction process, extraction of the files, can be achieved through normal zip extraction, something which is included by default in the Java version used for Android.
Once the files have been extracted, the application resources can be altered.

As explained earlier, both the application icon and the application name need to be altered to have a successful morphed application.
To alter the icons, first the original icon name that is stored in the resources is extracted from the application info, something which Android provides through its API.
Then it is a matter of iterating over the extracted content to find the appropriate icon files -- Android allows for multiple resolutions of the same image to be stored -- and replace them with the user specified icon.
Once the icons have been replaced, the name of the application needs to be changed.
Unfortunately, this is where the first restriction imposed by Android is encountered.

Android uses a resource map where each resource item gets mapped to a unique id generated by the compiler, which allows for easy re-use of resources in the making of an application.
This does pose an obstacle, since these reference maps are compiled into a binary format which is difficult to alter.
There is a solution available for desktop environments\cite{website:apktool}, but it relies on Android's Package Tool, `AAPT'.
Porting AAPT to Android proved to be near impossible due to its system requirements, which meant that decompiling the Android resources would not be a feasible approach.
Fortunately, the name of the application is accessed through a file known as the Android Manifest, which contains general information about the application package.
This manifest proved to be slightly more processable from within the confinements of Android.
This meant that the new application name could be put directly into the manifest, without having to rely on the decompiling of Android resources.
After these two steps of altering the application appearance, the contents can be reconstructed again into a valid Android Package.

The first step is rebuilding the archive. 
Since it is structurally the same as a Java Archive, existing tools could be used. 
For this, the JarBuilder by Dominik Werthmueller\cite{website:jarbuilder} was used, since it posed the least amount of dependencies, which is favorable when working with Android, which can be rather picky about what parts of Java are actually supported in its system.
This resulted in a complete archive, which still needed to be zipaligned and signed with an appropriate key.
Because of similar restrictions posed by the decompiling of the Android resources, it was decided that including zip alignment was not achievable for now.

Finally, the package needs to be signed. 
Fortunately, an existing standalone solution is available outside the default Android signing methods, the `zip-signer' library\cite{website:zip-signer}.
However, a signing key still needs to be chosen. 
For the scope of this project, it was decided the test key would be sufficient, since actually signing it with appropriate keys which would needed to be tracked to prevent falsification of the application provide several challenges which will be discussed in section \ref{sec:limitations:morphing}.

Once the archive has been signed the morphing has been completed. 
The user can now be presented with an application package which contains the original application, albeit with a new appearance, according to the user's preferences.
This application can then be shared with other users.




\subsection{The Encryption Service}
\label{sec:implementation:encryption}
DroidStealth uses a `Service'\footnote{An Android Service is a part of an application that runs in the background, often used for more intensive or lengthy executions.} for the encryption of files.
This service runs in the background, and listens for \texttt{Intents} started by the application.
A queue is used to order the requests, and the service running in the backgrounds works through the queue continuously.

The encryption that DroidStealth provides is implemented using Facebook's Conceal API\cite{facebookConceal}.
Conceal provides a set of APIs for data encryption and authentication; we only use the first.
The Conceal library does not implement cryptography algorithms, but instead uses algorithms used in OpenSSL\cite{openssl}.

The process of encrypting an unencrypted file is quite simple because of the use of the Conceal API:
The \texttt{Crypto} class, provided by the Conceal API, handles all encryption logic in the process.
A `plain' Java input stream is created from the unencrypted file, and the \texttt{Crypto} class provides an output stream that encrypts data as it is being passed through.
When both streams have been created, a dedicated algorithm copies the data, buffered in chunks of 4096 bytes, from the unencrypted file's input stream to the encrypting output stream.
The final result is, as expected, the encrypted version of the previously unencrypted file.
The decryption process uses the exact opposite method:
An input stream provided by the \texttt{Crypto} class provides the decryption logic, by which the encrypted file is read.
The output of that stream is passed to a plain Java output stream, which writes the data to a new, unencrypted file.

The files are then stored in a folder outside the application folder.
To allow the updating of the application without data loss, this separation is required; overwriting the application folder may be required, especially when installing a morphed version of DroidStealth.
This means that the folder storing the files is public; other applications, as well as the user, can find it on their device via a computer or with a file explorer app.
The risk posed by this is migitated by the encryption of the files, as well as the requirement to know of DroidStealth, transcending the scope of `casual search'.


\subsection{User Experience}
\label{sec:approach-and-design:user-experience}

\todo{Rewrite this since it's been moved here}

Within this subsection we will briefly elaborate on our design
choices in user experience design related to locking and hiding
of both data and the app. Finally, we briefly discuss the 
philosophy behind the styling. 

\subsubsection{Locking and Hiding of Data}

The most important part of this project, is the ability to hide
ones files. This means, as mentioned earlier, that hidden media
files should not be visible anymore in the Android media
gallery, and that with a file browser, one still shouldn't be
able to open any of the hidden files, as they would be
encrypted. However, this also means that while the sensitive
files are hidden and locked away, the user wouldn't be able to
directly open the files. Therefore, the user should have the
ability to unlock files, so that the data can be looked at,
shared, and even modified.

The issue here with unlocked files, is that these may be
forgotten by the user, and then, by a third party, found and
opened using a file browser. This issue was solved by showing
a clear warning to the user, explaining that some files are
unlocked and that, therefore, they could be leaked.
The user is presented with a persistent notification in the
notification drawer of Android, with the message that some of
the user's files are unlocked. When the notification is pressed,
all unlocked files are immediately locked. This
provides a user friendly way to swiftly lock the files back so
they can't be found, without the need to reopen the app, which
can take much more time, depending on the chosen method of
hidden and protecting the DroidStealth app. 

Furthermore, depending on the size, unlocking files could take a
minute or more, as the encryption service would have to perform a
full decryption process. During this process, another notification 
is shown, where the user is informed
of the fact that some files are currently in the process of
being unlocked. Once the user taps on it, the decryption is
canceled by emptying the queue; the currently de- or encrypted file is finished.


\subsubsection{Locking and Hiding of DroidStealth}

For an app that prevents others from finding your sensitive
data by means of a casual search, it is required that this app
can't be found easily as well. Obviously, if the app were to be
found by those you want to hide your files from, they will be
convinced that you might have sensitive files hidden. Hence, the
searcher might go further than a casual search, in order to
obtain the sensitive files. We have incorporated a few options to hide 
DroidStealth, and to launch the app using some secret methods.

\textbf{Casual Launch}
The trivial, default way of launching DroidStealth is through the app drawer of an Android device. 
When launching the application the first time, the user is promted to enter a pin that will be used to access the application.
On following launches, the app will present the user with a keyboard to enter the pin of the user. If the pin is correct, the 
DroidStealth will launch into the secret gallery. However, if the pin is incorrect, the user won't be able to enter the application.


\textbf{Hiding from App Drawer}
The user has the option to hide DroidStealth completely from the drawer. This 
means that the user should think of different means of launching the application, 
which will be discussed in the next parts. A limitation is that the app can still be found
in the application list in the Android settings screen. 

\textbf{Launch with Dialer}
The user may select the option to launch the application by calling a special number
in the regular dialer of the phone. Instead of actually calling the filled in number, the
application will be launched, presenting the user with the pin keyboard. Furthermore,
the application removes upon launch the last entry of the call log of the phone, so that
an attacked can't figure out the launch code, by simply checking the call log.

A potential security issue here, is that when a user fills in the wrong launch code, the 
entry won't be removed from the call log, and thus an attacked could be hinted towards
the right pin. Also, if a user had made more mistakes, the attacker could merge the 
suspicious call log entries, and deduct the correct pin. A possible solution would be to
check the call log for entries that are similar to the actual one, and to remove those as well.

\textbf{Launch with Widget}
Another option is to launch the app by means of an invisible widget on the user's home screen. 
When the user adds a widget to the home screen, it will still be visible, until the user presses on it. 
From that point on, when the user presses the invisible widget 5 times, the app is launched, showing the pin keyboard.

When adding a new access widget, all previously placed widgets become visible. This allows 
the user to retrieve forgotten widgets. Of course, this is a potential security threat, as this 
provides attackers an easy way of finding the hidden widget. We must remind ourselves, 
however, that we are focusing on casual search. Also, if this would be combined with
a morphed DroidStealth, attackers would never know which widget of which app should be
placed, unless widget previews are provided.



\subsection{Vulnerability Analysis}
\label{sec:vulnerability-analysis}
\todo{Expand and move sections here.}
DroidStealth has some vulnerabilities, which are analysed here.
Per vulnerability, we explain how it poses a risk, how big the risk is, and if applicable how this could be solved.

\subsubsection{Unlocked Files}
A potential security threat that can be exploited, is that
unlocked files are stored on the internal SD card of the device,
and that another app could constantly watch the folders in which
those files would be decrypted. Then this other app could save
them somewhere else or send them over the Internet.

Also, if files are unlocked, and the user's device is taken from
its rightful owner without warning, then the user might not have
gotten enough time to press the notifications to swiftly lock
the secret files back. Invaders won't be able to see the data
directly, unless they launch a file browser and start searching,
but they would be informed of the existence of secret files.
This could be a potential danger, because an attacker could be
interested in those files and do the user harm to get these
files. Users should be fully aware that they should only unlock
files if they are 100\% sure that they are in a safe location.

\todo{TODO:}

\todo{dialing wrong sequence: exposure}

\todo{installed application list in settings}



\section{Future Work}
\label{sec:future-work}

There are several aspects of the application than can be improved. Which
can be categorized under one of the following: usability, and morphing.
The categories will be discusses individually as they are separated both
technically and conceptually. 

\subsection{Usability}
In terms of usability there has been no formal study on what aspects of the application work work well. 
However, informal testing has shown several avenues of improvement. 

The testing has shown that the alternative launch methods could use closer attention when it comes to the usability.
The counterintuitive approaches to launching DroidStealth can be confusing.
There is also a risk that the methods, or the codes are forgotten when the user only sporadically opens the app;
This would render the data managed by the application lost forever.

With the providing of alternative access methods, easier access is provided to both the users and possible attackers.
The invisible widget that is provided is easily found in the widget list, though it is made more complex to detect when used in combination with a morphed DroidStealth.
The dialer launcher poses a risk when a user fills in the wrong launch code, as the entry won't be removed from the call log, and thus an attacked could be hinted towards the right pin. 
Also, if a user had made more mistakes, the attacker could merge the suspicious call log entries, and deduct the correct pin. 
A possible solution would be to check the call log for entries that are similar to the actual one, and to remove those as well.
Future work could address this issue with a solution that creates a more usable alternative launch method, while maintaining the security of the application.

Once the application has been launched, the user is required to pay close attention to the state of the application, and the state of the files managed by it.
A small human error can result in inadvertent data breaches, especially when the data in the application is opened in other apps.

Monitoring the application's lifecycle more closely could resolve some of these issues:
For example, `listening' to home button presses could be used to trigger the locking of files automatically whenever the app gets out of focus.
Another solution that has been discussed is limiting the time during which a file may be unlocked, which could restore the encryption after a (user-)designated duration.

\subsection{Morphing} 
\label{sec:limitations:morphing}

The morphing library has two major problems; app renaming restrictions and the lack of integrity verification.

Names used for morphing have to be the same length or shorter than the name originally given to the application.
Preliminary research indicates that this limitation can be overcome but not all size indicators in encoded Android manifests have not identified.
Alternatively increased support for self-compilation could remove this limitation.

The other component of naming is the `package name' staying consistent across morphs. 
The package name of an application is a naming convention which Android uses as an identifier for applications in its app store.
For casual searches, the package name is not relevant in most situations.
However there is one exception, namely if DroidStealth would be on an app store. 
As apps are identified there by package name this would make it easy to reveal its presence just by searching for it on the appstore.
Currently the only solution is to not be available through any app store.
\todo{move this paragraph to vulnerability analysis?}

Integrity validation is not included in morphing.
With integrity validation we mean that the actual code itself is not modified to change the behaviour of DroidStealth, for example introducing a backdoor.
Normally this is something that the Android package system takes care of, but morphing has unique demands that makes it impossible to rely on built-in measures.
These measures rely on using a private key for signing the code, however to allow updating for the application after morphing the private key has to be bundled with the software so it can be signed again.
With access to this key anyone can create a modified version of the morphed application and spread that as an `update'.
Solving this problem is non-trivial and requires more research.

\subsection{Reinstalling the application}
When the application is reinstalled from, for example, a morphed android application package, the encryption key used to lock data is replaced.
This means that the data locked before the reinstallation is impossible to decrypt; a significant limitation in the usability.

A method of deriving a key that is consistent through reinstalling the application is required, while it remains underivable for external parties.

Reinstalling the application from the newly created application package unfortunately replaces the encryption key that is used to encrypt and decrypt the files.
This means that the data stored before the reinstall cannot be unlocked anymore; a harsh limitation in the usga


\bibliographystyle{IEEEtran}
\bibliography{references}

\end{document}
