\documentclass[twocolumn,english,compsoc,journal]{IEEEtran}
\usepackage[T1]{fontenc}
\usepackage{babel}
\usepackage[unicode=true,
 bookmarks=true,bookmarksnumbered=true,bookmarksopen=true,bookmarksopenlevel=1,
 breaklinks=false,pdfborder={0 0 0},backref=false,colorlinks=false]
 {hyperref}
\hypersetup{pdftitle={Your Title},
 pdfauthor={Your Name},
 pdfpagelayout=OneColumn, pdfnewwindow=true, pdfstartview=XYZ, plainpages=false}
\usepackage{breakurl}

\makeatletter

%% Because html converters don't know tabularnewline
\providecommand{\tabularnewline}{\\}

 % protect \markboth against an old bug reintroduced in babel >= 3.8g
 \let\oldforeign@language\foreign@language
 \DeclareRobustCommand{\foreign@language}[1]{%
   \lowercase{\oldforeign@language{#1}}}

% for subfigures/subtables
\usepackage[caption=false,font=normalsize,labelfont=sf,textfont=sf]{subfig}
%\usepackage[nocompress]{cite} %optional

\makeatother

\begin{document}



\title{DroidStealth: A Nomadic Data Obfuscation Tool that Facilitates Sharing}


\author{Olivier Hokke, Alex Kolpa, Joris van den Oever, and~Alex Walterbos}


\markboth{Delft University of Technology Student Project}{Your Name \MakeLowercase{\textit{et al.}}: Your Title}

\IEEEtitleabstractindextext{
\begin{abstract}
TODO ABSTRACT TEXT\end{abstract}

\begin{IEEEkeywords}
casual search, privacy, nomadic software, obfuscation
\end{IEEEkeywords}

}

\maketitle

\IEEEdisplaynontitleabstractindextext{}


\IEEEpeerreviewmaketitle{}


\section{Introduction}

\IEEEPARstart{W}{ith} the rising use of smart phones in daily life and
exceptional events means sensitive data is commonly available on phones.
These pictures, videos, and in some cases other files are very interesting
to a wide variety of groups. And sometimes the standard protections on
the phone are not enough to keep those groups out. Just slide the screen
and full access to everything. Even if there things like a password or
unlock patterns these can be forced out of a person. Followed by a simple
search through the phones files by hand to find what they desire.

We call such non-technical searches of devices 'casual search.' The
person performing it will have varying degrees of expertise regarding
the workings of smartphones but is limited to the tools already available
on the phone to check out data. Instead of performing advanced attacks
using specialized tools designed to get all the data out of a phone.

This project aims to address only the issues with casual search through
the use of an Android app. The tool-aided expert attacks are not within the
scope of this project.

This paper starts with explaining the problem description in section \ref{sec:problem-description}, wherein the problem that led to the setup of this project is expanded upon.
Next, the approach taken to this problem is discusssed in section \ref{sec:approach-and-design}.
Once the approach has been determined, a description of the actual implementation can be provided, which is done is section \ref{sec:implementation}.
Finally, the limitations and future work will be discussed in section \ref{sec:future-work}.

\section{Problem Description}
\label{sec:problem-description}
Using mobile phones as a primary platform for computing, recording data and storing data is becoming more and more ubiquitous.
While the users store sensitive data on these devices the security measures to protect this data often prove to be insufficient.
This can be because of implementation failures but often these measures are hard to apply properly by the average user\cite{usablesecurity,cannotsecurity}.
Even the more popular security measures available, such as PIN codes, `unlock patterns' and passwords have proven not to be effective at all times.
There are even techniques to retrieve these codes and patterns, for example using smudge attacks\cite{aviv2010smudge}.

This section describes the problem we attempt to solve.
It does so by explaining the sensitive data that should be kept safe, the situation (environment) in which it should provide protection, and limitations that arise. 
First it will go into what kind of data it should involve and how to handle it. 
Then an explanation of what this data should be protected against. 
Followed by a section on 'root access' on a phone, what that means, and why that can not be required.
Finally we discuss the need for nomadic distribution.

\subsection{Hiding and Securing Data}
Sensitive data, files that the user wants to keep private, can take many forms.
The data will likely consist primarily of photos and videos with the occasional document but should not exclude other types of file.
As such the application should support the file types that the device could normally handle.

An obvious solution to keep data safe from undesired inspection is encrypting it.
Encryption, properly implemented, would provide a way of securing the data by making it impossible to access without the encryption key. 
This has the added benefit of hiding the encrypted data by making it unrecognizable as the file it once was.
Preventing many applications on a smart phone from showing them to a user, though not all.

Yet hiding the data is not enough.
The presence of the app can raise suspicion, especially when its purpose is known. 
This could result in follow-up measures.
Thus the app should be hard to find or recognize, further increasing the chance of passing a casual search.

\subsection{Casual Search}
As discussed in the introduction, we attempt to address the problem of `casual search'.
In the scope of this article, we define `casual search' as following: A quick attempt to find information on a mobile device without applying advanced technical knowledge of the device, nor specific knowledge of the protecting methods proposed and implemented as described in this article.

We aim to protect against this investigation method, as it is likely to be the first investigation encountered; pass it and a more thorough search will be unlikely.
Furthermore, casual search is probably the most common as more intensive searches usually require very limited resources.
After all a `quick glance' through a phone, be it at a checkpoint, a protest, or because your significant other has gotten hold of the device is more likely to occur than a thorough investigation of the device by someone who is experienced and trained where to look.

\subsection{No root}
The solution that is described in this article had a technical limitation that put a certain set of tools out of reach.
`Rooting' an Android device unlocks administrative privileges, allowing the user or applications to access restricted files, and communicate with device features more directly.
By doing this, it undermines Android's security model\cite{vidas2011all}.

Often named `root access', it can be compared with root access as in Unix systems, and Administrator privileges in Windows systems.
Because `root access' allows manipulating system files, it provides more powerful tools that could benefit the solution to the problem addressed.

However, most users do not have a rooted phone, nor do do they have the technical knowledge to root their phone themselves. 
Rooting a phone has significant disadvantages if done incorrectly; the device's software could get corrupted to the point where it needs to be completely reset or becomes unusable.
Users will also be hesitant to root their phone because it voids the warranty.
To keep the solution available for all Android phones, the use of root access is not an option, and can therefore not be applied.

\subsection{Nomadic distribution}
In situations where a central distributor of applications is monitored or censored, distribution of the application can prove difficult.
In these cases, the application we propose in our solution is likely to be blocked.
Even if it is not, knowledge of the application will quickly reach those who desire access to the data the users want to keep private.

A way of circumventing this could be a `nomadic' distribution of the application.
Instead of providing the application through a central point, that can be monitored, the users could share it directly with those interested.
More importantly, they can share it only with those they trust.
This provides extra means of keeping the application in the right hands, and thus preventing the knowledge of the application entirely.  

<<<<<<< HEAD
\section{Approach and Design}
\label{sec:approach-and-design}
This section walks though all design choices made with DroidStealth,
and how we approached the problem in order to tackle it. Most 
solutions have drawbacks, which will be discussed further in the
section about future work and limitations. %% ADD REFERENCE

%% Here we say stuff about the libraries and intents because why reinvent the wheel. 
%% WHAT ELSE DO WE TALK ABOUT HERE EXACTLY BESIDES USER EXPERIENCE?
%% NOT SURE IF STRUCTURE HERE WORKS

\subsection{Encryption}
\label{sec:approach-and-design:encryption}

\subsection{Morphing}
\label{sec:approach-and-design:morphing}

Simply hiding a user's data will not be enough to pass the casual search that was described earlier. 
Anyone inspecting a phone that knows of DroidStealth and its functionality would become highly suspicious of any user with the application installed, resulting in more thorough inspection. 
This is something this application was not designed for, which could possibly result in the hidden data becoming compromised. 

From this it follows that not only the data needs to be hidden, but DroidStealth as well. 
Unfortunately, the Android framework does not provide many tools for this. 
This is mainly because the operating system, like most mobile operating systems, is focused on easy application access instead of easy file access. 
Preventing detection meant some unconventional approaches had to be assessed.

There are several obstacles when it comes to staying unnoticed through a casual search on Android. 
The most obvious obstacle is the `app drawer,' which is an overview of all the applications installed on the Android device.
In Android is possible to change the way the app drawer behaves, since it is itself an application as well.
However, changing the app drawer is a radical approach, which many users, especially those not technically literate, would not consider doing.
Since the product was meant to be as easy as possible to install and use, it meant that having an application that behaved as a custom app drawer was not an option.

This meant that the application would have to be able to hide itself.
It would have to be so unnoticeable between the user's other applications, casual search would not expose it immediately. 
Where it led us to another problem, since applications have a name and an icon that stays the same as long as they are not updated by the developer themselves.
Updating continuously would not be an option, since DroidStealth is meant for offline usage, as explained in section \ref{sec:problem-description}. 

Since the developers can not consistently update the appearance of the application, this means it is something the users would need to be able to do themselves.
To achieve this, a novel approach was taken.
For this project we devised a method to allow users to alter the name and icon of DroidStealth without developer intervention, a method which was dubbed `morphing' due to the now fluid appearances DroidStealth.
Unfortunately, due to limitations in Android, users would need to re-install the application after having `morphed' it, but this problem is addressed through the way users share the application.

As expressed in the section \ref{sec:problem-description}, the application should be able to be shared nomadically, meaning there is no central distribution point, but users share the application between each other. 
Combined with the morphing discussed in the previous paragraph, this provides for a good solution for the necessity of reinstalling the application for it to be morphed.
Users can create a morphed version which they can share with other users, so that the application never looks the same, making it near impossible to detect during a casual search.

Sharing the application can be achieved through a multitude of ways on Android, but because of the ease of use and the increasing implementation, Near Field Communication (NFC)\cite{website:nfc-spec} was chosen the main approach for data sharing. 
It is also possible to share through the default sharing system supported by Android,which includes BlueTooth and uploading files to various sharing services, for those devices that do not yet support NFC.

\subsection{User Experience}
\label{sec:approach-and-design:user-experience}

Within this subsection we will briefly elaborate on our design
choices in user experience design related to locking and hiding
of both data and the app. Finally, we briefly discuss the 
philosophy behind the styling. 

\subsubsection{Locking and Hiding of Data}

The most important part of this project, is the ability to hide
files. This means, as mentioned earlier, that hidden media
files should no longer be visible in the media
galleries, and that with a file browser, one still shouldn't be
able to open any of the hidden files, as they would be
encrypted. While the sensitive
files are hidden and locked away, the user wouldn't be able to
directly open the files. Therefore, the user should have the
ability to unlock files, so that the data can be viewed,
shared, and even modified.

Unlocking files potentially causes security risks, they may be
forgotten by the user. Then later found by a third party when the user thought they were safe.
This issue was solved by showing
a clear warning to the user, explaining that some files are
unlocked and that this presents a risk.
The user is shown a persistent notification in the
notification drawer of Android, with the message that some of
the user's files are unlocked. When the notification is pressed,
all unlocked files are immediately locked. This
provides a user friendly way to swiftly lock the files again. Making it so
they can't be found without the need to reopen the app. As this
can take quite some time, depending on the chosen method of
hiding and protecting the DroidStealth app. 

Furthermore, depending on the size, unlocking files could take a
minute or more, as the encryption service would have to perform a
full decryption process. During this process, another notification 
is shown, where the user is informed
of the fact that some files are currently in the process of
being unlocked. Once the user taps on it, the decryption is
canceled by emptying the queue; the current de- or en-cryption is completed.


\subsubsection{Locking and Hiding of DroidStealth}

As described in the problem description the data and the app can't be easy to find.
Yet even as it is made to be hard to find it should still be easy to use.
While at first glance this may seem like contrary goals it does not have to be.
To this purpose we have incorporated several options to stealthily launch DroidStealth when it is hidden.

\textbf{Normal Launch}
The trivial, and default way of launching apps, is through the app drawer of an Android device. 
When launching the application the first time, the user is prompted to enter a pin that will be used to access the application.
On following launches, the app will present the user with a keyboard to enter the pin of the user. 
If the pin is correct, the DroidStealth will launch into the secret gallery.
However, if the pin is incorrect, the user won't be able to enter the application.
This pin will also be requested for all other launch methods. %Is this true?
This is the least obfuscating option, but not necessarily an insecure method.
Earlier we mentioned morphing DroidStealth to look like a different application.
When only morphing is used the resulting name and icon should not stand out from the other apps to prevent easy detection.

The user has the option to hide DroidStealth completely from the drawer. 
This means that the user is can  not turn on this feature unless other means of launching DroidStealth have been set up.
A listing of the other means will be discussed below.  %I think this might not be needed here? And the below is probably for the limitations section?
Important to note is that DroidStealth can still be found in the application list in the Android settings screen.
Meaning that morphing is still important even when the application is not in the app drawer.
As Morphing will change the icon and name for the application list.

\textbf{Launch with Dialer}
The user may select the option to launch the application by dialing a special phone number. 
Instead of actually calling the entered number, the application will be launched, presenting the user with the pin keyboard. 
DroidStealth ensures this number does not show up the call log of the phone when it is launched successfully.
This is so that an attacker can not find the phone number for launching by checking the call log.

%I think this section is meant for the limitations section later on too?
A potential security issue here is that when a user enters an incorrect phone number when trying to start DroidStealth.
The wrong number is not automatically removed from the call log. 
A smart attacker can potentially use this to shrink the search space for a brute force attack.
Though this is generally not feasible for a casual search it could still be an indicator for the presence of DroidStealth if invalid phone numbers are used.
This can potentially be solved would be to check the call log for entries that are similar to the actual one, and to remove those as well.

\textbf{Launch with Widget}
Another option is to launch DroidStealth by means of an 'invisible' widget on the user's home screen. 
When the user adds a widget to the home screen it will be visible.
Until the user presses on it at least, then it will go fully transparent.
Making it impossible to see under normal circumstances.
From that point on when the user presses the invisible widget 5 times DroidStealth is launched.

When adding a new access widget, all previously placed widgets become visible. 
This allows the user to retrieve widgets that have been misplaced.
To provide additional security this should be combined with a morphed DroidStealth.
Morphing obfuscates which widget of what app should be replaced unless widget previews are provided.

\subsubsection{Styling}
%Not sure if this fits right after bits about launch methods. And should those be under user experience if styling is there too?
%Also this says what we did and what it should do. But how do we know that? And was this tested? Maybe actually go into how things were tested?
%Add a bit about user experience testing being part of future work? Does this go across cultures too?
DroidStealth is themed with a dark color in order to give users the feeling that they are indeed
working in secret and that their data will be safe. However, DroidStealth should not
have a hobbyist feel to it. This would indicate that the code is potentially unstable, 
as it would give the user the sense that the developers were not serious or unqualified. 
Therefore, we used a rounded, but solid straight font for our titles, and a formal font for 
all other texts. Combining this with the use of bright, but slightly softened primary and 
secondary colors, the user is given the feeling that the app handles its tasks well, but in 
the mean time, the user is soothed with a comfortable look. %% TODO: ADD SMALL IMAGE

The use of green and red is mirrored from the conventional meaning of those colors. Green %How is that mirrored from the conventional meaning?
will always be used to indicate that something is good or safe, red will do the opposite. 
In addition we have the color orange, which is used to indicate progress. Whenever a file is 
locking, or unlocking, the status bar of the file will be orange, and a animated twirl is shown 
to indicate that the file is being processed. %% TODO: ADD SMALL IMAGE

Finally, the gallery uses no padding or margins between the thumbnails of files, as we want %Maybe mention android standards here?
to optimally make use of all the screen space. Thumbnails are thus as big as they can be, 
and thus will be most efficient in showing the user its contents. 

=======
\section{Approach and Design}
\label{sec:approach-and-design}
\todo{Add Two-Layer stuff}
This section walks though all design choices made with DroidStealth,
and how we approached the problem in order to tackle it. Most 
solutions have drawbacks, which will be discussed further in the
section about future work and limitations. %% ADD REFERENCE

%% Here we say stuff about the libraries and intents because why reinvent the wheel. 
%% WHAT ELSE DO WE TALK ABOUT HERE EXACTLY BESIDES USER EXPERIENCE?
%% NOT SURE IF STRUCTURE HERE WORKS

\subsection{Morphing}
\label{sec:approach-and-design:morphing}

\todo{Move elsewhere.}
Simply hiding a user's data will not be enough to pass a casual search. 
Anyone inspecting a phone that is aware of our application and its functionality would become highly suspicious of any user with the application installed, resulting in more thorough inspection. 
This is something this application was not designed for, which could possibly result in the hidden data becoming compromised. 

From this it follows that not only the data needs to be hidden, but the application itself as well. 
Unfortunately, the Android framework does not provide many tools for this. 
This is mainly because the operating system, like most mobile operating systems, is focused on easy application access instead of easy file access. 
This meant some unconvential approaches had to be assessed.

There are several obstacles in casual search for the application on Android. 
The most obvious obstacle is the so called `app drawer', which is an overview of all the applications installed on the Android device.
In Android is possible to change the way the app drawer behaves, since it in itself is an application as well.
However, changing the app drawer is a radical approach, which many users, especially those not technically literate, would not consider doing.
Since the product was meant to be as unobtrusive as possible to install and use, it meant that having an application that behaved as a custom app drawer was not an option.

This meant that the application would have to be able to hide itself.
It would have to be so unnoticable between the user's other applications, casual search would not expose it immediately. 
This proved another problem, since applications have a name and an icon that stays the same as long as they are not updated by the developer themselves. 
Updating continuously would not be an option, since the application is meant for offline usage, as explained in section \ref{sec:problem-description}. 

Since the developers can not update the appearance of the application, this means it is something the users would need to be able to do themselves.
To achieve this, a novel approach was taken.
For this project a method was devised to allow users to alter the name and icon of the application without developer intervention, a method which was appropriately dubbed `morphing' due to the now fluid behaviour application appearances can have.
Unfortunately, due to limitations in Android, users would need to re-install the application after having `morphed' it, but this problem is addressed through the way users share the application.

As expressed in the section \ref{sec:problem-description}, the application should be able to be shared nomadically, meaning there is no central distribution point, but users share the application between each other. 
Combined with the morphing discussed in the previous paragraph, this provides for a good solution for the necessity of reinstalling the application for it to be morphed: 
Users can create a morphed version which they can share with other users, so that the application never looks the same, making it near impossible to detect during a casual search.

Sharing the application can be achieved through a multitude of ways on Android, but because of the ease of use and the increasing implementation, Near Field Communication (NFC)\cite{website:nfc-spec} was chosen the main approach for data sharing. 
It is also possible to share through the default sharing system supported by Android -- which includes BlueTooth and uploading files to various sharing services -- for those devices that do not yet support NFC.

\subsection{Styling}

DroidStealth is themed with a dark color in order to give users the feeling that they are indeed
working in secret, and that their data will be safely hidden. However, the app should not
have an amateuristic feeling about it. This would indicate that the app is potentially unstable, 
since it would give the user the sense that the developers had no sense of perfection. 
Therefore, we used a rounded, but solid straight font for our titles, and a formal font for 
all other texts. Combining this with the use of bright, but slightly softened primary and 
secondary colors, the user is given the feeling that the app handles its tasks well, but in 
the mean time, the user is soothed with a comfortable look. %% TODO: ADD SMALL IMAGE

The use of green and red is mirrored from the conventional meaning of those colors. Green 
will always be used to indicate that something is good or safe, red will do the opposite. 
In addition we have the color orange, which is used to indicate progress. Whenever a file is 
locking, or unlocking, the status bar of the file will be orange, and a animated twirl is shown 
to indicate that the file is being processed. %% TODO: ADD SMALL IMAGE

Finally, the gallery uses no padding or margins between the thumbnails of files, as we want 
to optimally make use of all the screen space. Thumbnails are thus as big as they can be, 
and thus will be most efficient in showing the user its contents. 

>>>>>>> 7a830f7... Split user experience into own file, to implementation

\section{Implementation}
\label{sec:implementation}

This section will discuss the actual implementation of the approach as taken in section \ref{sec:approach-and-design}. 
It consists of the morphing implementation in section \ref{sec:implementation:morphing}, and the encryption implementation in section \ref{sec:implementation:encryption}.

\subsection{Morphing}
\label{sec:implementation:morphing}

As described in section \ref{sec:approach-and-design}, to hide the application some way is needed to alter the appearance of the application so that it can be hidden from casual search.
This section will detail the implementation of technique, which has been dubbed `morphing'.
To achieve a complete change of appearance for an application in the app drawer, both its icon and its name need to be changed. 
However, to explain the full approach, some background on the inner workings of Android application packages might be required.

Android uses its own naming for its packages, so called Android Packages, or `.apk' files.
In reality, these files are very similar to Java archives -- so called `.jar' files -- namely that they are both archives containing executables and resources.
To construct an apk file, one first needs the executables and the resources (images, animations and layouts among other things), which can then be compressed into an archive.
Optionally, the archive can be zipaligned, which improves its read performance by reordering the contents of the package as to optimize it for Android devices.
Once the archive has been constructed, it only needs to be signed with an appropriate key for the Android system to accept it as a valid package.

For the morphing to be successful, this process needed to be reversed, and then repeated after altering the archive resources.
For this, the original package is required. 
Fortunately, Android allows access to the original package from within the application without root access (see section \ref{sec:problem-description} for an explanation about what root access means).
Reversing of the package construction process, extraction of the files, can be achieved through normal zip extraction, something which is included by default in the Java version used for Android.
Once the files have been extracted, the application resources can be altered.

As explained earlier, both the application icon and the application name need to be altered to have a successful morphed application.
To alter the icons, first the original icon name that is stored in the resources is extracted from the application info, something which Android provides through its API.
Then it is a matter of iterating over the extracted content to find the appropriate icon files -- Android allows for multiple resolutions of the same image to be stored -- and replace them with the user specified icon.
Once the icons have been replaced, the name of the application needs to be changed.
Unfortunately, this is where the first restriction imposed by Android is encountered.

Android uses a resource map where each resource item gets mapped to a unique id generated by the compiler, which allows for easy re-use of resources in the making of an application.
This does pose an obstacle, since these reference maps are compiled into a binary format which is difficult to alter.
There is a solution available for desktop environments\cite{website:apktool}, but it relies on Android's Package Tool, `AAPT'.
Porting AAPT to Android proved to be near impossible due to its system requirements, which meant that decompiling the Android resources would not be a feasible approach.
Fortunately, the name of the application is accessed through a file known as the Android Manifest, which contains general information about the application package.
This manifest proved to be slightly more processable from within the confinements of Android.
This meant that the new application name could be put directly into the manifest, without having to rely on the decompiling of Android resources.
After these two steps of altering the application appearance, the contents can be reconstructed again into a valid Android Package.

The first step is rebuilding the archive. 
Since it is structurally the same as a Java Archive, existing tools could be used. 
For this, the JarBuilder by Dominik Werthmueller\cite{website:jarbuilder} was used, since it posed the least amount of dependencies, which is favorable when working with Android, which can be rather picky about what parts of Java are actually supported in its system.
This resulted in a complete archive, which still needed to be zipaligned and signed with an appropriate key.
Because of similar restrictions posed by the decompiling of the Android resources, it was decided that including zip alignment was not achievable for now.

Finally, the package needs to be signed. 
Fortunately, an existing standalone solution is available outside the default Android signing methods, the `zip-signer' library\cite{website:zip-signer}.
However, a signing key still needs to be chosen. 
For the scope of this project, it was decided the test key would be sufficient, since actually signing it with appropriate keys which would needed to be tracked to prevent falsification of the application provide several challenges which will be discussed in section \ref{sec:limitations:morphing}.

Once the archive has been signed the morphing has been completed. 
The user can now be presented with an application package which contains the original application, albeit with a new appearance, according to the user's preferences.
This application can then be shared with other users.




\subsection{The Encryption Service}
\label{sec:implementation:encryption}
DroidStealth uses a `Service'\footnote{An Android Service is a part of an application that runs in the background, often used for more intensive or lengthy executions.} for the encryption of files.
This service runs in the background, and listens for \texttt{Intents} started by the application.
A queue is used to order the requests, and the service running in the backgrounds works through the queue continuously.

The encryption that DroidStealth provides is implemented using Facebook's Conceal API\cite{facebookConceal}.
Conceal provides a set of APIs for data encryption and authentication; we only use the first.
The Conceal library does not implement cryptography algorithms, but instead uses algorithms used in OpenSSL\cite{openssl}.

The process of encrypting an unencrypted file is quite simple because of the use of the Conceal API:
The \texttt{Crypto} class, provided by the Conceal API, handles all encryption logic in the process.
A `plain' Java input stream is created from the unencrypted file, and the \texttt{Crypto} class provides an output stream that encrypts data as it is being passed through.
When both streams have been created, a dedicated algorithm copies the data, buffered in chunks of 4096 bytes, from the unencrypted file's input stream to the encrypting output stream.
The final result is, as expected, the encrypted version of the previously unencrypted file.
The decryption process uses the exact opposite method:
An input stream provided by the \texttt{Crypto} class provides the decryption logic, by which the encrypted file is read.
The output of that stream is passed to a plain Java output stream, which writes the data to a new, unencrypted file.

The files are then stored in a folder outside the application folder.
To allow the updating of the application without data loss, this separation is required; overwriting the application folder may be required, especially when installing a morphed version of DroidStealth.
This means that the folder storing the files is public; other applications, as well as the user, can find it on their device via a computer or with a file explorer app.
The risk posed by this is migitated by the encryption of the files, as well as the requirement to know of DroidStealth, transcending the scope of `casual search'.

<<<<<<< HEAD
=======

\subsection{User Experience}
\label{sec:approach-and-design:user-experience}

\todo{Rewrite this since it's been moved here}

Within this subsection we will briefly elaborate on our design
choices in user experience design related to locking and hiding
of both data and the app. Finally, we briefly discuss the 
philosophy behind the styling. 

\subsubsection{Locking and Hiding of Data}

The most important part of this project, is the ability to hide
ones files. This means, as mentioned earlier, that hidden media
files should not be visible anymore in the Android media
gallery, and that with a file browser, one still shouldn't be
able to open any of the hidden files, as they would be
encrypted. However, this also means that while the sensitive
files are hidden and locked away, the user wouldn't be able to
directly open the files. Therefore, the user should have the
ability to unlock files, so that the data can be looked at,
shared, and even modified.

The issue here with unlocked files, is that these may be
forgotten by the user, and then, by a third party, found and
opened using a file browser. This issue was solved by showing
a clear warning to the user, explaining that some files are
unlocked and that, therefore, they could be leaked.
The user is presented with a persistent notification in the
notification drawer of Android, with the message that some of
the user's files are unlocked. When the notification is pressed,
all unlocked files are immediately locked. This
provides a user friendly way to swiftly lock the files back so
they can't be found, without the need to reopen the app, which
can take much more time, depending on the chosen method of
hidden and protecting the DroidStealth app. 

Furthermore, depending on the size, unlocking files could take a
minute or more, as the encryption service would have to perform a
full decryption process. During this process, another notification 
is shown, where the user is informed
of the fact that some files are currently in the process of
being unlocked. Once the user taps on it, the decryption is
canceled by emptying the queue; the currently de- or encrypted file is finished.


\subsubsection{Locking and Hiding of DroidStealth}

For an app that prevents others from finding your sensitive
data by means of a casual search, it is required that this app
can't be found easily as well. Obviously, if the app were to be
found by those you want to hide your files from, they will be
convinced that you might have sensitive files hidden. Hence, the
searcher might go further than a casual search, in order to
obtain the sensitive files. We have incorporated a few options to hide 
DroidStealth, and to launch the app using some secret methods.

\textbf{Casual Launch}
The trivial, default way of launching DroidStealth is through the app drawer of an Android device. 
When launching the application the first time, the user is promted to enter a pin that will be used to access the application.
On following launches, the app will present the user with a keyboard to enter the pin of the user. If the pin is correct, the 
DroidStealth will launch into the secret gallery. However, if the pin is incorrect, the user won't be able to enter the application.


\textbf{Hiding from App Drawer}
The user has the option to hide DroidStealth completely from the drawer. This 
means that the user should think of different means of launching the application, 
which will be discussed in the next parts. A limitation is that the app can still be found
in the application list in the Android settings screen. 

\textbf{Launch with Dialer}
The user may select the option to launch the application by calling a special number
in the regular dialer of the phone. Instead of actually calling the filled in number, the
application will be launched, presenting the user with the pin keyboard. Furthermore,
the application removes upon launch the last entry of the call log of the phone, so that
an attacked can't figure out the launch code, by simply checking the call log.

A potential security issue here, is that when a user fills in the wrong launch code, the 
entry won't be removed from the call log, and thus an attacked could be hinted towards
the right pin. Also, if a user had made more mistakes, the attacker could merge the 
suspicious call log entries, and deduct the correct pin. A possible solution would be to
check the call log for entries that are similar to the actual one, and to remove those as well.

\textbf{Launch with Widget}
Another option is to launch the app by means of an invisible widget on the user's home screen. 
When the user adds a widget to the home screen, it will still be visible, until the user presses on it. 
From that point on, when the user presses the invisible widget 5 times, the app is launched, showing the pin keyboard.

When adding a new access widget, all previously placed widgets become visible. This allows 
the user to retrieve forgotten widgets. Of course, this is a potential security threat, as this 
provides attackers an easy way of finding the hidden widget. We must remind ourselves, 
however, that we are focusing on casual search. Also, if this would be combined with
a morphed DroidStealth, attackers would never know which widget of which app should be
placed, unless widget previews are provided.


>>>>>>> 7a830f7... Split user experience into own file, to implementation

\section{Future Work}
\label{sec:future-work}

There are several aspects of the application than can be improved. Which
can be categorized under one of the following: usability, and morphing.
The categories will be discusses individually as they are separated both
technically and conceptually. 

\subsection{Usability}

In terms of usability there has been no formal study on what aspects of
the application work work well. However informal testing has shown several
avenues of improvement. First is how to access the application. 

While the freely chosen app name and icon do suffice there are other ways 
it could be done. Namely the widgets available to android users. This could 
be a simple widget mimicking the standard available widgets, or even be 
invisible. So as to have no recognizable screen space. While it could still
react to certain use patterns. A first implementation of this functionality is
available but needs to be properly evaluated.

Once the application has been started there are still some ways the user
needs to be observant in how she uses the application. As it can result in
inadvertent data breaches. Mainly when the data is opened in other apps.
While several solutions have been discussed, like listening to home button
presses or having files only be unlocked for limited time, these still need
to be explored in depth.

\subsection{Morphing} 
\label{sec:limitations:morphing}

When it comes to morphing there are two major limitations; app renaming and software integrity validation. 
The first pertains to limitations in the renaming of the application when morphing. 
The second holds that there is no way to guarantee that the application hasn't been modified after morphing.

When it comes to naming there is a functional limitation in the length of the new name as of now.
It has to be the same length or shorter than the original name length, which can be padded with whitespace. 
But as of yet the way Android manifests are encoded are opaque enough that not all size indicators have been correctly identified.

The other component of naming is the `package name' staying consistent across morphs. 
The package name of an application is a naming convention which Android uses as an identifier for applications in its app store.
For casual searches, the package name is not relevant in most situations.
However there is one exception, namely if DroidStealth would be on an app store. 
As apps are identified there by package name this would make it easy to reveal its presence just by searching for it on the appstore.
Currently the only solution is to not be available through any app store.

The integrity validation is currently something that does not exist with morphing. 
The application needs to be signed with the same key before morphing as after morphing if users want to be able to reinstall it afterwards without first having to remove it. 
This makes it possible to distribute updates in a nomadic manner, but also means the signing key has to be packaged with DroidStealth for it to work. 
As such anyone can create a modified version of the application and spread that as an `update.'
Solving this problem is non-trivial and requires more research.

\subsection{Security}
A potential security threat that can be exploited, is that
unlocked files are stored on the internal SD card of the device,
and that another app could constantly watch the folders in which
those files would be decrypted. Then this other app could save
them somewhere else or send them over the Internet.

Also, if files are unlocked, and the user's device is taken from
its rightful owner without warning, then the user might not have
gotten enough time to press the notifications to swiftly lock
the secret files back. Invaders won't be able to see the data
directly, unless they launch a file browser and start searching,
but they would be informed of the existence of secret files.
This could be a potential danger, because an attacker could be
interested in those files and do the user harm to get these
files. Users should be fully aware that they should only unlock
files if they are 100\% sure that they are in a safe location.

\section*{Acknowlegments}

The authors would like to thank Dhr JA Pouwelse from the Delft University of Technology for providing the possibility to investigate this matter, for providing his helpful insights and feedback, and placing his expertise at our disposal.

\bibliographystyle{IEEEtran}
\bibliography{references}

\end{document}
