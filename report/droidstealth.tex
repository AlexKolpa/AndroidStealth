\documentclass[twocolumn,english,compsoc,journal]{IEEEtran}
\usepackage[T1]{fontenc}
\usepackage{babel}
\usepackage[unicode=true,
 bookmarks=true,bookmarksnumbered=true,bookmarksopen=true,bookmarksopenlevel=1,
 breaklinks=false,pdfborder={0 0 0},backref=false,colorlinks=false]
 {hyperref}
\hypersetup{pdftitle={Your Title},
 pdfauthor={Your Name},
 pdfpagelayout=OneColumn, pdfnewwindow=true, pdfstartview=XYZ, plainpages=false}
\usepackage{breakurl}

\makeatletter

%% Because html converters don't know tabularnewline
\providecommand{\tabularnewline}{\\}

 % protect \markboth against an old bug reintroduced in babel >= 3.8g
 \let\oldforeign@language\foreign@language
 \DeclareRobustCommand{\foreign@language}[1]{%
   \lowercase{\oldforeign@language{#1}}}

% for subfigures/subtables
\usepackage[caption=false,font=normalsize,labelfont=sf,textfont=sf]{subfig}
%\usepackage[nocompress]{cite} %optional

\makeatother

\begin{document}



\title{DroidStealth: A Nomadic Data Obfuscation Tool that Facilitates Sharing}


\author{Olivier Hokke, Alex Kolpa, Joris van den Oever, and~Alex Walterbos}


\markboth{Delft University of Technology Student Project}{Your Name \MakeLowercase{\textit{et al.}}: Your Title}

\IEEEtitleabstractindextext{
\begin{abstract}
TODO ABSTRACT TEXT\end{abstract}

\begin{IEEEkeywords}
casual search, privacy, nomadic software, obfuscation
\end{IEEEkeywords}

}

\maketitle

\IEEEdisplaynontitleabstractindextext{}


\IEEEpeerreviewmaketitle{}


\section{Introduction}

\IEEEPARstart{W}{ith} the rising use of smart phones in daily life and
exceptional events means sensitive data is commonly available on phones.
These pictures, videos, and in some cases other files are very interesting
to a wide variety of groups. And sometimes the standard protections on
the phone are not enough to keep those groups out. Just slide the screen
and full access to everything. Even if there things like a password or
unlock patterns these can be forced out of a person. Followed by a simple
search through the phones files by hand to find what they desire.

We call such non-technical searches of devices 'casual search.' The
person performing it will have varying degrees of expertise regarding
the workings of smartphones but is limited to the tools already available
on the phone to check out data. Instead of performing advanced attacks
using specialized tools designed to get all the data out of a phone.

This project aims to address only the issues with casual search through
the use of an Android app. The tool aided expert attacks are not within the
scope of this project. As this is a very different kind of problem.

\textbf{Bla bla bla structure of paper and what we're going to discuss
where.}


\section{Problem Description}
\label{sec:problem-description}
Using mobile phones as a primary platform for computing, recording data and storing data is becoming more and more ubiquitous.
While the users store sensitive data on these devices the security measures to protect this data often prove to be insufficient.
This can be because of implementation failures but often these measures are hard to apply properly by the average user\cite{usablesecurity,cannotsecurity}.
Even the more popular security measures available, such as PIN codes, `unlock patterns' and passwords have proven not to be effective at all times.
There are even techniques to retrieve these codes and patterns, for example using smudge attacks\cite{aviv2010smudge}.

This section describes the problem we attempt to solve.
It does so by explaining the sensitive data that should be kept safe, the situation (environment) in which it should provide protection, and limitations that arise. 
First it will go into what kind of data it should involve and how to handle it. 
Then an explanation of what this data should be protected against. 
Followed by a section on 'root access' on a phone, what that means, and why that can not be required.
Finally we discuss the need for nomadic distribution.

\subsection{Hiding and Securing Data}
Sensitive data, files that the user wants to keep private, can take many forms.
The data will likely consist primarily of photos and videos with the occasional document but should not exclude other types of file.
As such the application should support the file types that the device could normally handle.

An obvious solution to keep data safe from undesired inspection is encrypting it.
Encryption, properly implemented, would provide a way of securing the data by making it impossible to access without the encryption key. 
This has the added benefit of hiding the encrypted data by making it unrecognizable as the file it once was.
Preventing many applications on a smart phone from showing them to a user, though not all.

Yet hiding the data is not enough.
The presence of the app can raise suspicion, especially when its purpose is known. 
This could result in follow-up measures.
Thus the app should be hard to find or recognize, further increasing the chance of passing a casual search.

\subsection{Casual Search}
As discussed in the introduction, we attempt to address the problem of `casual search'.
In the scope of this article, we define `casual search' as following: A quick attempt to find information on a mobile device without applying advanced technical knowledge of the device, nor specific knowledge of the protecting methods proposed and implemented as described in this article.

We aim to protect against this investigation method, as it is likely to be the first investigation encountered; pass it and a more thorough search will be unlikely.
Furthermore, casual search is probably the most common as more intensive searches usually require very limited resources.
After all a `quick glance' through a phone, be it at a checkpoint, a protest, or because your significant other has gotten hold of the device is more likely to occur than a thorough investigation of the device by someone who is experienced and trained where to look.

\subsection{No root}
The solution that is described in this article had a technical limitation that put a certain set of tools out of reach.
`Rooting' an Android device unlocks administrative privileges, allowing the user or applications to access restricted files, and communicate with device features more directly.
By doing this, it undermines Android's security model\cite{vidas2011all}.

Often named `root access', it can be compared with root access as in Unix systems, and Administrator privileges in Windows systems.
Because `root access' allows manipulating system files, it provides more powerful tools that could benefit the solution to the problem addressed.

However, most users do not have a rooted phone, nor do do they have the technical knowledge to root their phone themselves. 
Rooting a phone has significant disadvantages if done incorrectly; the device's software could get corrupted to the point where it needs to be completely reset or becomes unusable.
Users will also be hesitant to root their phone because it voids the warranty.
To keep the solution available for all Android phones, the use of root access is not an option, and can therefore not be applied.

\subsection{Nomadic distribution}
In situations where a central distributor of applications is monitored or censored, distribution of the application can prove difficult.
In these cases, the application we propose in our solution is likely to be blocked.
Even if it is not, knowledge of the application will quickly reach those who desire access to the data the users want to keep private.

A way of circumventing this could be a `nomadic' distribution of the application.
Instead of providing the application through a central point, that can be monitored, the users could share it directly with those interested.
More importantly, they can share it only with those they trust.
This provides extra means of keeping the application in the right hands, and thus preventing the knowledge of the application entirely.  

\section{Design and software architecture}

Here we say stuff about the libraries and intents because why reinvent
the wheel.

Data sharing and application sharing.


\section{Implementation}

How did we create the data vault. Encryption, thumbnails, notifications
for unlocking.

What the hell do we do to make morphing possible.


\section{Future Work}

There are several aspects of the application than can be improved. Which
can be categorized under one of the following: usability, and morphing.
The categories will be discusses individually as they are separated both
technically and conceptually. 

\subsection{Usability}

In terms of usability there has been no formal study on what aspects of
the application work work well. However informal testing has shown several
avenues of improvement. First is how to access the application. 

While the freely chosen app name and icon do suffice there are other ways 
it could be done. Namely the widgets available to android users. This could 
be a simple widget mimicking the standard available widgets, or even be 
invisible. So as to have no recognizable screen space. While it could still
react to certain use patterns. A first implementation of this functionality is
available but needs to be properly evaluated.

Once the application has been started there are still some ways the user
needs to be observant in how she uses the application. As it can result in
inadvertent data breaches. Mainly when the data is opened in other apps.
While several solutions have been discussed, like listening to home button
presses or having files only be unlocked for limited time, these still need
to be explored in depth.

\subsection{Morphing}

When it comes to morphing there are two major limitations, app renaming
and software integrity validation. The first pertains to limitations in the renaming
of the application when morphing. The second holds that there is no way
to guarantee that the application hasn't been modified after morphing.

When it comes to naming there is a functional limitation in the length
of the new name as of now.
It has to be the same length or shorter than the original name length,
which can be padded with whitespace. But as of yet the way android
manifests are encoded are opaque enough that not all size indicators
have been correctly identified.

The other component of of naming is the package name staying consistent
across morphs. This is not relevant for casual search in most situations
however there is one exception, ff the application is on an appstore. As
apps are identified there by package name this would make it easy to
reveal the presence of the app. Just by searching for it on the appstore.
Currently the only solution is to not be available on appstores.

The integrity validation is currently something that does not exist with
morphing. The application needs to be signed with the same key
before morphing as after morphing. This makes it possible to distribute
updates in a nomadic manner. But also means the signing key is packaged
with the application itself for it to work. As such anyone can create a
modified version of the application and spread that as an 'update.'
Solving this problem is non-trivial and requires more research.

However initial exploration of this issues has pointed us in the direction
of a second signer application that verifies the code based on other
indicators, e.g. hash. This would then produce the application and
preferably uninstall itself. As this would allow the signer app to be
on a appstore without revealing the presence of the application. Then 
all updates would have to happen through instances of this application.
This all needs to be properly designed and verified however.

\section*{Acknowlegments}

The authors would like to thank Dhr JA Pouwelse from the Delft University of Technology for providing the possibility to investigate this matter, for providing his helpful insights and feedback, and placing his expertise at our disposal.

\bibliographystyle{IEEEtran}
\bibliography{references}

\end{document}
